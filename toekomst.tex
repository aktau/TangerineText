\chapter{Toekomstig werk}
\label{toekomst}

\section{Nieuwe platformen}
Met dit project is ook een verdere stap gezet naar mobiele toepassingen. Het data-luik staat bijvoorbeeld toe om applicaties voor tablets te schrijven die beroep kunnen doen op een externe database om een groot deel van het zware werk over te nemen. Dit soort programma's kunnen het manueel verifi\"eren van paren sneller en aangenamer maken. De huidige werkwijze voor finale verificatie is als volgt: nadat er een resem waarschijnlijke paren zijn ge\"identificeerd, kan men de kisten met fragmenten uit de opslagruimte halen en nakijken welke er \'echt passen. Gezien de grote hoeveelheid brokstukken is het niet mogelijk om ze allemaal bij de hand te houden. Daardoor duurt het altijd even voor de gewenste fragmenten gevonden worden. In het slechtste geval wordt er gewerkt op een (krachtige) desktop. Hierdoor is het nodig om ofwel de namen en locaties van de fragmenten te onthouden, ofwel een heleboel afbeeldingen af te drukken. Na het fysisch testen van de fragmenten moet men dan terug naar de desktop om de bevindingen in te geven. Gelukkig behoort een laptop ook tot de mogelijkheden hoewel applicaties als Browsematches en Griphos niet bepaald licht zijn. Het performantieprobleem wordt natuurlijk reeds een deel verholpen door het invoegen van een externe database. Ook kan men nu gemakkelijker met meerdere mensen en laptops tegelijkertijd werken aan de validaties wegens de automatische synchronisatie. Een stap verder zou zijn om een tablet te gebruiken. Er zijn reeds in het verleden experimenten geweest binnen het thera project om aanraakgevoelige omgevingen te maken en de hoop is dat tesamen met de resultaten van deze thesis er in de toekomst iets concreets van gemaakt kan worden.\\

\section{Verder innoveren}
Dynamische herberekening met procesverloop. In de realiteit wordt de finale conclusie over een voorstel gemaakt als men de fysieke brokstukken op de aangegeven plaats aan elkaar zet en ze ``klikken''. Door erosie is de klik op zich niet altijd 100\% vast waardoor het gebeurt dat zelfs met de fragmenten in de hand een amateur nog altijd geen sluitende conclusie kan geven. Evenwel is het \'e\'en van de krachtigste controles en zijn er niet veel voorstellen die twijfelgevallen blijven. Misschien is (een deel van de) oplossing om de weergave dynamisch te maken en zo te pogen de stabiliteit van het voorstel te meten. Uit ervaring is gebleken dat mensen snel kunnen opmerken wanneer een voorstel \textbf{potentieel} heeft, maar niet in een oogopslag kunnen beslissen over de juistheid. Vaak blijft er zelfs na een betere kijk op de statische virtuele beelden nog genoeg twijfel over: noch de gewone visualisatie nog de beschikbare alternatieven geven uitsluitsel. In acht nemend dat de automatische passers van het thera project rekening moeten houden met computationele effici\"entie en daarom niet te granulair kunnen zoeken~\cite{Brown2008}, is er dikwijls nog wat ruimte voor verbetering. Een manier om dit op te merken is dat de doorsneden van paren die correct blijken te zijn vaak nog delen vertonen waar de volumes van beide brokstukken elkaar snijden. De erosie die plaatsvindt op de fresco's die onderzocht worden is echter puur subtractief, er zet zich geen extra materiaal vast op de brokstukken. Er is dus een goede kans dat een juist voorstel verder kan geoptimaliseerd worden.\\

Interessante voostellen zouden daarom kunnen gebruik maken van een adaptieve versie van de passer die, beginnend van de originele positie, heel granulair kan zoeken. Wat voor miljoenen combinaties te kostelijk is, is voor een enkel voorstel slechts een peulenschil. Dit proces van iteratieve verbetering kan getoond worden aan de gebruiker, door een tijdsverloop van de doorsnede van het paar tijdens de optimalisatie weer te geven. Het komen en gaan van gebieden waar de geometrie van de fragmenten snijdt en waar ze verder uit elkaar gaat kan op zich reeds informatief zijn, maar het eindpunt is cruciaal. Na een volledige optimalisatie kan er een beeld getoond worden van lichte verschuivingen van het optimale punt. De hypothese is dat --- als de resolutie van de 3D-opname voldoende is --- de doorsnede van een paar dat in de realiteit klikt herkenbare patronen zal vertonen als het verschoven wordt. Hierover valt te verstaan dat er weerstand zal zijn in de vorm van snijdende geometrie. Er wordt hier abstractie gemaakt van enkele belangrijke implementatiedetails, zoals de richting waarin verschoven wordt. Dit kan gaan van eenvoudigweg de richting parallel kiezen aan de scheidingslijn tussen de uiterste punten, tot een complexe fysische simulatie die de fragmenten als rigide lichamen voorstelt en een combinatie van drukkende en schuivende krachten uitoefent. In de eerste vorm kan de weerstand ``gemeten'' worden door de gebieden van snijdende geometrie te sommeren. De tweede vorm laat toe de frictie onrechtstreeks te meten aan de hand van bijvoorbeeld verwachte tegenover re\"ele snelheid in een bepaalde richting, of beter nog de kracht die nodig was om een verschuiving te krijgen. Het is voorlopig onduidelijk of de toegenomen complexiteit van een fysische simulatie in verhouding staat met de resultaten.\\

Natuurlijk kan een aspect van deze beweging ook opgenomen worden als een karakteristiek in lerende algoritmen, men zou het ``stabiliteit'' kunnen noemen. Dit is een voorbeeld van een eigenschap die als discriminator kan gebruikt worden op voorstellen die reeds goed of interessant te noemen zijn, een slecht voorstel zou namelijk veel stabiliteit kunnen vertonen door de vele intersecties die zelfs na optimalisatie overblijven (afhankelijk van de precieze methode waarop men stabiliteit meet). Er zijn ook goede voorstellen waar de stabiliteit laag kan zijn doordat het materiaal teveel is afgevlakt of de breuk gewoonweg te rechtlijnig is. De stabiliteit lijkt dus geen gouden graal te zijn. Daarom wordt hoe langer hoe duidelijker dat net zoals een mens meerdere eigenschappen in rekening brengt bij het oordelen, een computer dit ook moet doen.
