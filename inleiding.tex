\chapter{Inleiding}
\label{inleiding}
Het reconstrueren van fresco's waarvan in opgravingen fragmenten gevonden worden is een moeilijke taak. Men kan het vergelijken met het oplossen van een enorme puzzel waarvan de stukken arbitraire vormen hebben, de meesten hun originele kleur zijn verloren en er vele anderen ontbreken. Daarbovenop ondervinden veel fragmenten erosie over de eeuwen heen, waardoor ze niet meer perfect op elkaar passen en confirmatie nog moeilijker wordt.\\

\begin{figure}[ht]
	\begin{center}
		\includegraphics[width=0.6\columnwidth]{images/WDC7_26.JPG}
		\caption{Een bak vol fragmenten die rond dezelfde locatie zijn gevonden, sommige zijn reeds aan elkaar gezet.}
		\label{fig:bakinleiding}
	\end{center}
\end{figure}

De stukken met een nog zichtbaar geometrisch patroon of van de rand van het fresco zijn in vergelijking met de anderen eenvoudig met elkaar te verbinden. Zij zijn door een ervaren archeoloog zonder hulp in elkaar te passen. De overige fragmenten die minder informatie bevatten zijn echter een nachtmerrie om aan elkaar te puzzelen. Er ontbreekt informatie die een structuur vanop een afstand laat zien, het menselijke visuele systeem is blijkbaar niet erg geschikt enkel grillige randen te vergelijken en aan elkaar te zetten. Het enige alternatief lijkt om lokaler te zoeken en elk fragment te vergelijken met elk ander fragment. Deze aanpak is natuurlijk niet mogelijk, er zijn miljoenen combinaties van fragmenten mogelijk.\\

In deze context situeert zich het thera\footnote{Thera is de oude naam voor het huidige griekse eiland genaamd Santorini, waar het project voor het eerst in de praktijk werd toegepast.} project, dat probeert om het werk van de archeoloog gemakkelijker te maken door middel van een software platform~\cite{Brown2008}. De
redenering achter het project is dat een computer de ondankbare taak voorgesteld in de vorige paragraaf kan automatiseren. Dergelijk systeem werd in 2007 aan de \emph{Princeton} universiteit in Amerika geconcipieerd. Sindsdien is er tot op de dag van vandaag door verschillende onderzoekers van over de hele wereld aan gewerkt. De werking van het systeem wordt in het volgende hoofdstuk nader toegelicht.\\

Deze thesis draait rond het maken van een uitbreiding op het platform. De uitbreiding moet de gebruikers van het systeem in staat stellen om de beschikbare data op nieuwe manieren te gebruiken, te visualiseren, aan te passen en te delen met medeonderzoekers. Aangaande terminologie zullen een paar woorden veelvuldig terugkomen: de termen fragment, brokstuk of gewoonweg stuk verwijzen altijd naar een enkel gebroken deel van het originele fresco. De termen fragmentpaar, paar en voorstel zijn gereserveerd voor een aaneenkoppeling van twee fragmenten op een bepaalde plaats. Het valt te benadrukken dat twee fragmenten met elkaar verschillende paren kunnen vormen, indien zij op verschillende punten raken (ook al liggen die punten niet zo ver uit elkeaar). Om deze reden wordt een paar ook vaak een voorstel genoemd, om te benadrukken dat het een mogelijke configuratie is, maar geen zekere. Verder verwijst deze tekst meermaals naar (automatische) herkenners, dit zijn de identificatiealgoritmen die twee fragmenten aan elkaar proberen passen.\\

\section{Overzicht}
Hoofdstuk~\ref{hoofdstuk:overzicht} geeft een overzicht van het bestaande werk en hoofdstuk~\ref{hoofdstuk:doelen} geeft aan op welke vlakken het project zal uitgebreid worden en waarom. Vervolgens behandelt hoofdstuk~\ref{hoofdstuk:ontwerp} het algemene ontwerp van de applicatie. Hoofdstukken~\ref{hoofdstuk:database},~\ref{hoofdstuk:synchronisatie} en~\ref{hoofdstuk:modules} gaan dieper in op enkele specifieke implementaties van het project zoals de database, de synchronisatie en de modules. Tenslotte volgt een besluit in hoofdstuk~\ref{hoofdstuk:besluit}.
