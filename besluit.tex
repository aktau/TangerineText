\chapter{Besluit}
\label{besluit}

Dit thesisproject heeft een lange weg afgelegd, waarbij er vaak geen duidelijk doel te bespeuren was, of beter gezegd geen concreet doel (vaag vaag).

Net zoals een gebroken fresco in feite een puzzel is, kan men het thera project zien als de som van vele delen die in elkaar passen. Dit thesisproject is bedoeld als een stuk dat een ander perspectief biedt op het geheel en het in staat moet stellen om meer en sneller resultaten te boeken. Het complementeert de bestaande aanpakken en zorgt ervoor dat de resultaten van de automatische paarherkenning nog nuttiger gebruikt kunnen worden. Omdat het validatiewerk noodzakelijk door mensen moet gebeuren, kan dit niet zonder meer opnieuw door een algoritme gedaan worden indien er iets misloopt. De in dit thesisproject geproduceerde componenten proberen er onder andere voor te zorgen dat dit zo min mogelijk voorkomt.\\

Op het vlak van ontginning van nuttige informatie met nieuwe visualisaties en nieuwe manieren om de juiste patronen te ontdekken zijn er natuurlijk nog steeds vele opportuniteiten. Want --- zoals opgemerkt in een recente paper over het thera project [citatie siggraph submission 2011] --- het vinden van de juiste paren is zoals zoeken naar een naald in een hooiberg. Met elke nieuwe toevoeging aan de mogelijkheden van het platform is er de kans dat deze een manier is om de hooiberg te verkleinen, door te lichten met X-stralen of gewoonweg op een grote krachtige magneet in een windtunnel te plaatsen. Naar deze laatste methode is iedereen natuurlijk op zoek. Tot dan is het zeker belangrijk dat men gemakkelijk kan experimenteren alsook bijhouden en opvragen welk deel van de berg reeds doorkamt is, waar de gevonden naaldrijke aders zitten en wat hun eigenschappen zijn. Wie weet zijn de naalden immers niet van metaal\ldots