\chapter{Bespreking van het database ontwerp}
\label{hoofdstuk:database}
De visuele delen van het platform bekommeren zich met de \textbf{de manier waarop} er data moet weergegeven worden, de gebruiker kiest \textbf{wat} er moet weergegeven worden. Om de data hiervoor aan te voeren is er een datalaag nodig die elk verzoek juist en effici\"ent kan verzorgen.

\section{Vereisten}
De hoofdpunten waaraan deze laag moet voldoen werden reeds opgesomd in hoofdstuk~\ref{hoofdstuk:doelen}, namelijk: snelheid, universele toegang, synchroniseerbaarheid en maximale flexibiliteit voor data-analyse. Eerst wordt een analyse gemaakt van de structuur van de data. Vervolgens wordt er gekeken naar de huidige methode die het thera project hiervoor hanteert. Enkele alternatieven worden besproken en op basis daarvan wordt een keuze gemaakt.

\subsection{Ontleding van de data}
De kern van de data is een verzameling fragmentparen, deze bezitten elk een paar kernattributen en vele optionele. De kernattributen beperken zich tot de namen (of een identificatienummer) van de fragmenten, en een transformatiematrix die het \'ene brokstuk aan het andere koppelt, er is geen enkel paar dat deze gegevens mist. Verder kunnen de herkenningsalgoritmen en eventueel post-processingalgoritmen allerlei soorten data toevoegen als attributen. Dit kan gaan om een maatstaf afhankelijk van het pasproces zoals de zogenaamde \emph{RibbonMatcherError}\cite{Brown2008} of een algemene eigenschap zoals het overlappende volume, in hoofdzaak gaat het hier om decimale en re\"ele getallen. Verder kan een gebruiker eigen data toevoegen, zoals een validatie of commentaar bij een voorstel.

\subsection{Nuttige data die nog niet bestaat}
\subsubsection{Gebruikers}
Behalve paren en attributen is het in een collaboratieve omgeving handig om gebruikers bij te houden. Omdat het systeem in eerste instantie voor selecte groepen is, is een gebruikersnaam en een mailadres voldoende. Op die manier zou men kunnen vastleggen wie welk stukje data heeft toegevoegd of veranderd. Dit is nuttig om de oorzaak van problemen op te sporen als die zich voordoen en te beslissen wiens data voorrang krijgt als eenzelfde fragment door een andere persoon gewijzigd werd (zie synchronisatie).

\subsubsection{Geschiedenis}
Synchronisatie is een vereiste, hetgeen detectie en resolutie van conflicten impliceert. Detectie kan automatisch uitgevoerd worden door na te kijken of de huidige inhoud van een veld verschillend is. Resolutie daarentegen is niet altijd volledig te automatiseren. Om de gebruiker werk te besparen tijdens het synchroniseren moeten er toch zoveel mogelijk conflicten zonder manuele invoer opgelost worden. Cruciaal om dit te verwezenlijken is een geschiedenis van elk attribuut, dit wil zeggen een lijst die aanduidt op welk moment, door wie en naar welke waarde een attribuut is veranderd. Het bijhouden van een geschiedenis heeft ook andere voordelen, zo kan bijvoorbeeld de geschiedenis van een ``commentaar''-attribuut gebruikt worden als een manier om te corresponderen. De overgang van validaties van een paar (bijvoorbeeld \emph{niet geweten} $\rightarrow$ \emph{misschien} $\rightarrow$ \emph{niet correct} of  \emph{niet geweten} $\rightarrow$ \emph{misschien} $\rightarrow$ \emph{correct}) kan een nuttige bron van data zijn voor zowel mensen (statistieken) als computers (lerende algoritmen).

// merging: three way merge + context-informatie! (welke statussen beter zijn et cetera\ldots)
// een simplificatie die veel moeilijkheden vermijd bij het synchronizeren waar versiecontrolesystemen voor broncode hun tanden op breken, is dat in dit geval er geen verwijderingen kunnen gebeuren. Het huidige algoritme zal nooit een paar of attribuut verwijderen, er wordt ook geen geschiedenis bijgehouden voor de toevoeging of verwijdering van paren.

  
...relationeel van aard? (de geschiedenis wel, denormalisatie nodig voor NoSQL) => SQL database
Het Eigenschap-patroon (Property pattern)

Huidige situatie = Properties pattern? % ( http://steve-yegge.blogspot.com/2008/10/universal-design-pattern.html#redacted )

\section{Het oude systeem: XML-bestanden}
De fragmentparen en hun attributen worden door automatische herkenners opgeslagen in een XML bestand zoals in figuur~\ref{code:fragxml}. Dit formaat is uitermate geschikt voor het overbrengen van de data naar andere subsystemen en is leesbaar door een mens. Het is minder geschikt als een permanent opslag- en zoekformaat. Dat laatste is echter de manier waarop het nu gebruikt wordt, met trage applicaties tot gevolg. Om een redelijke snelheid te behouden tijdens het zoeken of sorteren moeten Griphos en Browsematches het XML bestand volledig in het werkgegheugen plaatsen. Voor een document met 50000 paren neemt dit reeds een goede 300MB in beslag voor paren met elk 11 attributen (zonder afbeeldingen of 3D-modellen). Dit extrapoleren naar een miljoen paren geeft 5GB aan vereist werkgeheugen, hetgeen bij de meeste systemen van vandaag leidt tot het wegschrijven van data naar het bestandsssyteem of eerder nog een geheugenallocatiefout.\\

Het inladen van dit XML bestand is tijdrovend. Tabel~\ref{table:matchesloadspeed} laat zien dat het inladen minutenlang kan duren. Griphos en Browsematches gebruiken dezelfde manier om data in te lezen, maar vertonen desalniettemin een groot verschil in laadtijd. De reden is dat Griphos meteen ook de afbeeldingen van elk fragment ophaalt. Meer nog, na het laden van 50000 paren neemt het Griphos proces 1.5GB RAM in beslag. Het lijkt erop dat dit geheugen niet vrijgemaakt noch herbruikt wordt, want een tweede keer de lijst met fragmentparen openen zorgde er op het testsysteem met 2GB RAM voor dat Griphos automatisch werd afgesloten wegens overallocatie. Browsematches (en de thesisapplicatie) laden de afbeeldingen pas wanneer ze eigenlijk nodig zijn en besparen op die manier veel geheugen. Het thesisproject gaat eigenlijk nog een stapje verder en laadt zelfs de fragmenten pas in wanneer ze nodig zijn, dit komt later aan bod. De reden dat de tijd dan niet gewoon 0 seconden aangeeft is omdat er steeds moet geteld worden hoeveel voorstellen er in totaal aanwezig zijn, dit kost bij sommige databaseimplementaties soms wat tijd.

\begin{table}[h]
	\rowcolors{2}{gray!50}{white}
	\begin{center}
		\begin{tabular}{|l|r|r|r|}
		    \rowcolor{gray!75}
		    \hline
		    & \textbf{Griphos} &  \textbf{Browsematches} & \textbf{Thesis} \\
		    \hline
		    \textbf{\textasciitilde 4000 paren laden} & 54 sec & 16 sec & 0 sec \\
		    \textbf{\textasciitilde 50000 paren laden} & 7 min 18 sec & 2 min 47 sec  & 0-3 sec* \\
		    \textbf{\textasciitilde 250000 paren laden} & niet getest & niet getest & 0-15 sec* \\
		    \hline
		\end{tabular}
	\end{center}
	\caption{Meting van de tijden die elk programma nodig heeft om een collectie paren in te laden}
	\label{table:matchesloadspeed}
\end{table}

* De tijden voor de oplossing van de thesis vari\"eren afhankelijk van de achterliggende databaseimplementatie en hoeveel data er nog in het geheugen geladen is. De maximale tijden kwamen bij tests enkel voor net na het opstarten van de database, de gemiddelde laadtijd is ongeveer 100 milliseconden.

Eenmaal geladen, is het niet zo eenvoudig om te navigeren in deze collectie

Als het echter aankomt op het aanwenden van de data om in te zoeken schiet het formaat tekort.

\lstinputlisting[float=h,label=code:fragxml,caption=Uittreksel van een fragmentpaarbestand (hier worden slechts 2 paren getoond)]{source/shortmatches.xml}

XML bestand volledig in het geheugen ingeladen ---> 300 MB voor 50000 paren, dus voor 2 miljoen paren\ldots

E\'en van de belangrijkste stappen op weg naar een collaboratieve applicatie, is het omvormen van\\

Het ontwerp van het database model en de implementatie details zijn groot genoeg om hun eigen hoofdstuk te verdienen, zij worden later in hoofdstuk [...] besproken.\\

\section{Alternatieven}
Het moet gratis zijn! (niet overdreven veel middelen in het thera project en ``enterprise'' oplossingen zijn vaak zeer prijzig) ==> open-source of tenminste gratis.

NoSQL (Cassandra, \ldots) http://arin.me/blog/wtf-is-a-supercolumn-cassandra-data-model --- http://nosql.mypopescu.com/post/573604395/tutorial-getting-started-with-cassandra (statische sort == slecht voor onze doeleinden)
SQL
XML database (oud) (zie wiki) XQuery, \ldots

XML dismissed -> volgende sectie is kiezen tussen SQL en NoSQL

Redenering!

\subsubsection{Paren}

\subsubsection{Attributen van paren}

\subsubsection{Complexe informatie die niet in een simpel attribuut past}

[uitleggen wat duplicates zijn, mss in ander hoofdstuk, mss in het stuk van de modules!]
[comments = correspondentie enzovoort]

\subsection{Pagination}

--> beperken van datatransfer
--> het niet steeds opnieuw vragen van hoeveel fragmenten er voldoen aan de criteria

\section{SQL of niet}
Er is dus een idee van wat er moet opgeslagen worden. Er is een grote vari\"eteit aan oplossingen mogelijk

De term NoSQL is eigenlijk niet zo goed gekozen (waarom?), maar wordt hier gebruikt omdat het de facto de standaard manier is om naar niet-relationele databases te verwijzen.

De zogenaamde ``NoSQL'' oplossingen schieten als paddenstoelen uit de grond. De meerderheid hiervan functioneren als sleutel-waarde opslag. Door de veelheid aan oplossingen en hun (relatief) jonge leeftijd is het niet evident een volledige vergelijking te maken. Nogal wat artikels over NoSQL oplossingen lijken zonder meer te evangeliseren, waardoor het moeilijk is een realistisch beeld te krijgen van hun specifieke sterktes en zwaktes.

NoSQL, lichte uitleg over NoSQL (Cassandra, \ldots)

De belofte van object serialisatie en schemaloze\footnote{In een traditionele relationele database is steeds een schema aanwezig dat precies specificeert wat een object (= rij in een tabel) is. Het is minder eenvoudig om de definitie van object aan te passen door bijvoorbeeld een attribuut (= kolom) toe te voegen, maar niet onmogelijk.} opslag is verleidelijk. De basisvereisten van dit thesisproject qua data gaan in eerste aanleg niet verder dan het opslaan/serialiseren van objecten (\emph{User}, \emph{FragmentPair}, \emph{History}). Het niet weten welke attributen er in de toekomst gaan belangrijk zijn lijkt goed te passen in het schemaloze aspect.\\

Een goede vergelijking van beide aanpakken kan als er gekeken wordt naar: kan de implementatie het soort operaties dat nu nodig is aan (op performante wijze? Kan de implementatie voorziene/niet-vooziene uitbreidingen aan (many-to-many)?\\


Vergelijking van NoSQL tegenover SQL oplossingen waar die toepasselijk zijn op de vereisten van het project

Berkeley DB, Tokyo Cabinet, CouchDB, MongoDB, Cassandra

Berkeley DB en Tokyo Cabinet zijn lightgewight en snelle oplossingen die er echter op vertrouwen dat alle data in het werkgeheugen past, de ondersteuning voor complexe vragen is ook bijna onbestaande. 

Er zijn vele argumenten over wat wel en niet geschikt is voor elk type database, waarvan een heel deel tegenstrijdig zijn. Uit dit kluwen komt eigenlijk slechts \'e\'en goede raad naar voren: probeer het zelf. Het is echter niet zo evident noch de beste besteding van tijd om twee verschillende maar complete subsystemen uit te bouwen om te zien welke de bovenhand haalt. Omdat er nog verschillende gebruiksscenarios niet gekend zijn werd er gekozen voor een SQL oplossing, waarvan de auteur meer zekerheid had dat het de taak en alle toekomstige taken aankon.

\rowcolors{2}{gray!50}{white}
\begin{center}
	\begin{tabular}{|l|r|r|r|}
	    \rowcolor{gray!75}
	    \hline
	    & \textbf{SQL} &  \textbf{NoSQL} & \textbf{Kyoto Cabinet} \\
	    \hline
	    \textbf{Portabiliteit/Uniformiteit} & Hoog (gestandaardiseerde taal) & Laag (veel verschillende talen) \\
	    \textbf{Ondersteuning in de toekomst} & Hoog & Laag (vele projecten die een onzekere levensduur hebben) \\
	    \hline
	\end{tabular}
\end{center}

Beide systemen bezitten voord- en nadelen. Wat echter duidelijk is, is dat bij NoSQL databases er op voorhand moet geweten zijn welk type query het meest frequent gaat zijn (Cassandra\ldots)
 Omdat er geen 100\% zekerheid is over de uitbreidingen die gaan komen aan het datamodel

Beide modellen voor dataopslag kunnen in principe dezelfde soort operaties aan, de verschillen bevinden zich in de gemakkelijkheid van uitvoeren en de performantie. Het is echter nog steeds niet evident welke uiteindelijk de optimale keuze is. De wereld van het databeheer bevindt zich in een stroomversnelling en op elk gegeven moment kan er een fantastische nieuwe oplossing uit de bus komen. Het valt op te merken dat op de manier er nu van een XML database gemigreerd zal worden naar een andere technologie, dit later openieuw kan gebeuren als er een duidelijk betere oplossing zich voordoet. Alle operaties op de objecten die in dit project reeds zijn ge\"implementeerd, zijn vanuit het perspectief van de gebruiker --- archeoloog of ontwikkelaar --- database-agnostisch. De schijnbare uitzondering op deze regel lijkt het filteren te zijn, waar een valide (vereenvoudigde\footnote{Hoewel er niet gecontroleerd wordt of er geen SQL subqueries in de modelfilter aanwezig zijn, is het gebruik ervan niet ondersteund (hoewel het voor de meeste SQL-databases geen probleem is deze filters uit te voeren).}) SQL WHERE gebruikt wordt om condities op de eigenschappen van paren te zetten. Deze zijn echter niet zo complex (zie \ref{code:sortingfiltering}) en dus gemakkelijk om te bouwen naar iets wat een willekeurige onderliggende database kan begrijpen. De WHERE-syntax kan dus ook, ook al verstaat de database het niet zonder meer. Indien dit niet genoeg is kunnen er gewoon meerdere filtertypes zijn, een SQL-type, een CouchDB-type, enzovoort. Een implementatie van een model zal dan zelf moeten uitzoeken hoe het alles moet vertalen of gewoon de filter weigeren.

Niet alleen is objectpersistentie nodig (de sterkte van key-value NoSQL databases), maar de manier om de objecten te vinden moet zo flexibel mogelijk zijn. NoSQL databases bieden verschillende hulpmiddelen aan om dit te doen (MapReduceCombine, CouchDB Views), \ldots 

Een lichte voorkeur voor de SQL oplossingen is niet verwonderlijk. Er zijn verschillende implementaties met uniforme taal De auteur is reeds bekend met de werking en als men moeilijk kan voorspellen wat voor een queries zullen nodig zijn (data-mining) blijken SQL databases sneller te zijn. SQL lijkt gemakkelijker te gebruiken.\\

Een van de belangrijkste voordelen van NoSQL type databases is de ingebouwde schaleerbaarheid over meerdere machines. Het kost werkelijk geen moeite om deze implementaties over meerdere machines te laten samenwerken en hun performantie gaat zo goed als lineair met het aantal machines omhoog. Zullen de datavereisten van het thera project ooit zo groot worden dat een enkele krachtige machine niet meer voldoet? Op het eerste zicht lijkt het van niet, hedendaagse . Daarenboven is het daarom niet onmogelijk om SQL databases in een cluster te laten werken, gratis open-source implementaties als MySQL en PostgreSQL hebben beide ondersteuning voor clustering~\cite{postgrescluster, mysqlcluster}. Het verschil is de eenvoudigheid waarmee dit uit te bouwen is.\\

De syntax van NoSQL is niet simpeler, ook niet voor de specifieke queries om paren op te halen (zie infographic). Een andere vaststelling is dat de bibliotheek die door het thera project en dus ook dit thesisproject gebruikt wordt, Qt, ingebouwde ondersteuning heeft voor SQL databases. Het nalezen van artikels en conversaties op het internet en private correspondentie met ervaringsdeskundigen heeft geleerd dat NoSQL databases hun snelheid laten blijken bij veel transacties en een beperkte klasse van verzoeken, die moet echter op voorhand gekend zijn\cite{cassandradatamodel}. In essentie moet de data ingedeeld worden zodat het soort verzoeken dat verwacht wordt snel kan afgehandeld worden [REFERENTIE]. Hoewel een deel van de verzoeken die nodig zijn in dit project daadwerkelijk gekend zijn, is \'e\'en van de uitgangspunten net dat er een grote vari\"eteit aan verzoeken moet mogelijk zijn.  

Om al de redenen is er gekozen om het stabiele SQL pad te kiezen, met de deur open naar eventuele alternatieven wanneer die een overtuigend voordeel bieden.

[figuur mogelijke NoSQL uitbreiding, TokyoMatchModel, CouchDBModel,\ldots] 

Een voordeel van SQL systemen is dat er eenvoudig tussen verschillende imlementaties kan gemigreerd worden als de vereisten veranderen. Een broekzakversie die bij momenten geen connectie met het internet heeft kan op het compacte SQLite vertrouwen terwijl een krachtige server een heel grote database kan draaien met MySQL/PostgreSQL/Oracle/\ldots. Al deze verschillende implementaties kunnen met dezelfde subsystemen aangesproken worden. Bij vele NoSQL systemen is dit soort flexibiliteit niet mogelijk (een uitzondering is bijvoorbeeld http://fallabs.com/kyotocabinet/).\\

http://stackoverflow.com/questions/5438500/example-of-a-task-that-a-nosql-database-cant-handle-if-any (auto-sync\ldots)
http://stackoverflow.com/questions/2403174/is-there-any-nosql-database-as-simple-as-sqlite
Supergoede argumenten voor en tegen NoSQL/SQL: http://buytaert.net/nosql-and-sql

Dat gezegd zijnde, behalve de query interface

\section{Eerste iteratie}

Zonder geschiedenis/gebruikers

Geschiedenis invullen kan met triggers, maar dit is op het moment niet zo, op die manier heeft een applicatie de vrijheid om veranderingen aan te brengen zonder de geschiedenis te vervuilen. Als dit een slechte beslissing blijkt te zijn kunnen de triggers wel aangezet worden.

\subsection{De database laag}

UML-achtig schema met de Database, het MatchModel, SQLFragmentConf, de interopabiliteit!!!!

----> hieruit vloeit de requirement van een matches tabel en attributen

Om compatibel te zijn met het grotere thera project moeten alle paren hun eigen eigenschappen kunnen aanpassen. Dit wordt in de SQL versie opgelost door te eisen dat valide paren steeds geconstrueerd worden met een referentie naar de database waar ze uit komen. Op deze manier wordt de meest eenvoudige implementatie van een paar ongeveer zo geschreven:

\lstinputlisting[label=fragmentpair1,caption=Een versimpelde interface voor een FragmentPair en diens implementatie met directe verzoeken aan de database, language=C++]{source/pairinterface.cpp}

\subsection{Een grote tabel of vele verschillende}

Om de attributen van een paar te modelleren, kan er gekozen worden tussen deze allemaal op te slaan in de hoofdtabel (\emph{matches}) of voor elk attribuut een andere tabel aan te maken.

Indien het mogelijk zou zijn dat een (grote) subset van paren een attribuut gewoonweg niet bezit, kan het 

Hybride aanpak:

History altijd in verschillende tabellen

%http://stackoverflow.com/questions/4056093/what-are-the-disadvantages-of-using-a-key-value-table-over-nullable-columns-or
%http://stackoverflow.com/questions/695752/product-table-many-kinds-of-product-each-product-has-many-parameters
%http://en.wikipedia.org/wiki/Entity-Attribute-Value_model

Zijn de meeste attributen gedefini\"eerd voor alle paren?

Voordelen denormalizatie: de index wordt niet ettelijke keren redundant herhaald
Voordelen normalizatie

Simpele attributen, database query voor elk attribuut (dit bleef in stand tot er op een externe server geprobeerd werd)
Directe write-through is nooit een echt probleem omdat dit minder vaak gebeurt, op minder fragmenenten\ldots men zal veel vaker van venster wisselen dan ineens alle statussen omgooien

Dependency analyse voor filters

\subsection{Complexe informatie voorstellen als een attribuut}
Meta-attributen, VIEWs ---> duplicates (conflicts?)

\subsection{Geschiedenis}
De geschiedenis van een attribuut bijhouden heeft twee voordelen: het is nodig voor samenvoegalgoritmen + het is een weer een bron van data (welke paren zijn veel van status veranderd?) + sommige attributen worden op die manier nuttiger, 

\section{Verschillende database systemen}

\subsection{SQLite}
Aan de start van het project leek

\subsection{MySQL}
MySQL werd eerst gekozen wegens de alomtegenwoordige

\subsection{De late toevoeging van PostgreSQL}
Dankzij de flexibiliteit die nodig was om zowel SQLite als MySQL het grootst mogelijke deel van de code te laten delen

\section{Data mining}
Maak ketting van (yes+maybe) ---> zoek naar alle buren binnen 1 hop (algoritme is geschreven)

\section{Externe database traag, interne database snel}
Het grote probleem dat duidelijk werd na het rechtstreeks werken met externe databases, is dat zelfs op een lokaal netwerk de verzoeken een zeer grote vertraging opleverden.\\

De redenen hiervoor waren natuurlijk de vele queries die elk object kan versturen om zijn attributen op te halen of te veranderen.\\

Nog een groot voordeel bij het maken van dergelijk systeem is dat er gebruik kan gemaakt worden van niet-desctructieve veranderingen. Op het einde van een werksessie kan men de volledige
lokale database ofwel in de hoofd database invoegen ofwel gewoon verwijderen.\\

Het design moest transparant zijn, zowel een rechstreekse connectie als een gebufferde connectie zouden voor de eindgebruiker en de ontwikkelaar aan de buitenkant hetzelfde lijken.\\

Idee: Vervang object ID met object hash voor snellere match-to-match identificatie zonder een globala id nodig te hebben

\section{Slimme client of slimme server?}
Het programma bevat alle nodige functionaliteit en de server is in dat opzicht gewoon de specifieke database waarnaar het een verwijzing heeft.

Voor het gebruik op kleine apparaten (zoals tablets) zijn de mogelijkheden voor computationeel zware activiteiten niet zo uitgebreid. Daarenboven moet ervoor gezorgd worden dat de batterijduur
van dit apparaat niet te hard beknot wordt door het gebruik van de applicatie. Daarom zal de transitie naar zeer mobiele platformen enkel mogelijk worden indien er een simpele client applicatie kan
ontwikkeld worden die op de server vertrouwd om de juiste berekeningen te maken.\\

In de thesisperiode is geen tijd gevonden om dit concept uit te werken maar er zijn wel plannen gemaakt waardoor dergelijke
applicatie op een effici\"ente manier zou kunnen ontwikkeld worden. De gebruikersinterface Tangerine voert alle functies met betrekking tot de achterliggende data uit met behulp van een grote bibliotheek
van klassen die op zich niets met de interface te maken hebben. Er zou dus een alternatief programma gebouwd kunnen worden die in plaats van met een grafische interface kan bediend worden
via een zelfgemaakt protocol. Dit soort ontwerp wordt vaak aangetroffen in de UNIX wereld, waar er een enkele server-variant van een programma bestaat en meerdere mogelijke clients die ervan gebruik maken.
Het bekendste is misschien wel de X server.

Materialized views!!!

Model change batching!!! (startBatch/endBatch)

Incompatibiliteiten tussen *SQL

Analyse van datatoegangspatroon: vooral SELECT, ORDER BY, GROUP BY ---->
veelvuldig gebruik van indexes

Metadata preloading (na het testen van de snelheid van het ophalen van metadata
over een internetconnectie, werd besloten om\ldots)

De gevaren van een stale cache! (en ook de gevaren van een stale window, ook
een soort cache)). Oplossing altijd een request sturen om te dubbelchecken?!
---> te traag

optimize voor fast reads -----> inserts kunnen lokaal gedaan worden, updates hopelijk niet zo veel of lokaal

Recheck om de X seconden: doenbaar indien materialized views met indices!
(eigenlijke window query < 1 msec). Systeem zou 1000'en gebruikers kunnen
ondersteunen, zeker met een grotere cache

Cache = write-through

Vele discplines van de computerwetenschappen

% http://en.wikipedia.org/wiki/Three-way_merge#Three-way_merge <--- (bij design\ldots)

dynamisch zoekopdrachten genereren

SQLite formaat maakt database gemakkelijk te delen (USB-stick) <-- geen internet
connectie

Detecteren van imcompatibiliteiten EN veranderingen met reguliere
expressies:

Aanpaken voor syncrhonizatie:
moeilijk: changelogging functionaliteit, triggers, \ldots (mogelijk maar
moeilijk cross-db en error-prone)
``gemakkelijk'': Maak gewone queries zeer goedkoop

Vereisten: Scaleerbaarheid (XML schaalt NIET)
Voordeel van XML is echt wel dat het een mooi outputformaat is voor
matchers\ldots import capaciteit voorzien ==> import naar temp SQLite db en
merge

Ondersteuning voor ``meerdere snelheden'', minder modules hebben is niet erg,
elke additie is uitbreiding. Zo ook minder/geen versie van db-schema problemen

dependency scanning!

\section{Benchmarking}
De query cache staat af
Om de variabiliteit van de filesystem (nederlands) cache een beetje buiten spel te zetten zijn al deze routines ``opgewarmd''

Pagination
Late row lookup

Ideaal = minder dan een seconden voor elke gegeven query vanuit een gebruikersinteractie standpunt (System Response Time and User Satisfaction pagina 5)

Effect van DB configuratie (veel geheugen\ldots)

Suggested workaround voor het text probleem -> sphinx, restrict fragment names (niet ZO gemakkelijk), string + nummer
Suggested workaround voor het indexing probleem (zoals gezien voor status IN (\ldots)) -> force een index?! dunno\ldots hij pakt in ieder geval de verkeerde!

'High Performance MySQL', Second Edition, O'REILLY, ISBN: 978-0-596-10171-8
MySQL Reference Manual for version 5.1

http://nlp.stanford.edu/IR-book/html/htmledition/permuterm-indexes-1.html (dit is hoe wildspeed werkt) (LIKE performance lijkt niet zo slecht in Postgres, het is de sorting eerder\ldots)
http://www.slideshare.net/techdude/how-to-kill-mysql-performance
http://stackoverflow.com/questions/1540590/how-to-speed-up-like-operation-in-sql-postgres-preferably <--- use trigrams (fail), MAAR BETER IN 9.1 (future research)
https://cgsrv1.arrc.csiro.au/blog/2010/06/23/materializedindexed-views-for-postgresql/