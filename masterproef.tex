\documentclass[master=cws]{kulemt}
\setup{title={Een uitbreidbaar platform voor het effici\"{e}nt reconstrueren van fresco's},
  author={Nicolas Hillegeer},
  promotor={Prof.\,dr.\,ir.\ P. Dutr\'{e}},
  assessor={Prof.\,dr.\ D. De Schreye\and Dr.\ A.~Lagae},
  assistant={Dr.\ B.J.~Brown}}

% De volgende \setup mag verwijderd worden als geen fiche gewenst is.
\setup{filingcard,
  translatedtitle={An extendible platform for the efficient reconstruction of frescoes},
  udc=621.3,
  shortabstract={Het reconstrueren van archeologische fresco's is een domein waar computers nuttig werk kunnen verrichten, zowel voor archiveringsdoeleinden als voor digitale reconstructie. De algoritmen voor het automatisch ontdekken van passende fragmenten zijn reeds erg geavanceerd, evenwel moeten alle resulterende paarvoorstellen nog steeds verwerkt worden. Huidige tools zijn krachtig maar missen integratie, gebruiksvriendelijkheid en uitbreidbaarheid. Dit eindwerk tracht hierin verandering te brengen door het ontwikkelen van een platform dat doorgedreven analyse en manipulatie van voorstellen toelaat. Het resultaat is een applicatie die toelaat om verschillende personen vanop een afstand aan eenzelfde collectie te laten werken. Netwerktoegankelijke data zorgt ervoor dat de meest actuele informatie steeds beschikbaar is, terwijl de synchronisatie erover waakt dat er geen waardevolle vondsten verloren gaan. Tegenover het vroegere systeem is de sterk uitgebreide analysecapabiliteit ook nuttig, aangezien er hiermee complexe vragen over de collectie kunnen gesteld worden. Als laatste is het mogelijk om snel modules te maken die nieuwe visualisaties toelaten en deze eenvoudig aan te koppelen op het datasysteem. Enkele voorbeelden werden reeds gemaakt voor dit eindwerk.}}


\setup{coverpageonly}
%\setup{frontpagesonly}

% Kies de fonts voor de gewone tekst, bv. Latin Modern
\setup{font=lm}

% Hier kun je dan nog andere pakketten laden of eigen definities voorzien
\usepackage[dutch]{babel}
\usepackage{url}
%\usepackage[pdfusetitle,colorlinks,plainpages=false,urlcolor=blue]{hyperref}
\usepackage{pdfsync}
\usepackage[final]{pdfpages}
%\usepackage[draft]{pdfpages}
\usepackage{amsmath}
\usepackage{graphicx}
%\setkeys{Gin}{draft} % force draftmode
\usepackage{subfig}
\usepackage{hvfloat}
\setlength{\parindent}{0in}

\usepackage{listings}
\usepackage{courier}
\lstset{
 		 %identifierstyle=\ttfamily,
         %keywordstyle=\color[rgb]{0,0,1},
         commentstyle=\color[rgb]{0.133,0.545,0.133},
         %stringstyle=\color[rgb]{0.627,0.126,0.941},
         stringstyle=\color{blue},
         %stringstyle=\ttfamily\color[rgb]{0.627,0.126,0.941},
 		 columns=fullflexible,
         basicstyle=\footnotesize\sffamily, % Standardschrift  \footnotesize\ttfamily \small\sffamily
         %numbers=left,               % Ort der Zeilennummern
         numberstyle=\tiny,          % Stil der Zeilennummern
         %stepnumber=2,               % Abstand zwischen den Zeilennummern
         numbersep=5pt,              % Abstand der Nummern zum Text
         tabsize=2,                  % Groesse von Tabs
         extendedchars=true,         %
         breaklines=true,            % Zeilen werden Umgebrochen
         %keywordstyle=\color{red},
    	 frame=b,
         keywordstyle=[1]\textbf,    % Stil der Keywords
         keywordstyle=[2]\textbf,    %
         keywordstyle=[3]\textbf,    %
         keywordstyle=[4]\textbf,   %\sqrt{\sqrt{}} %
         %stringstyle=\color{white}\ttfamily, % Farbe der String
         showspaces=false,           % Leerzeichen anzeigen ?
         showtabs=false,             % Tabs anzeigen ?
         xleftmargin=17pt,
         framexleftmargin=17pt,
         framexrightmargin=5pt,
         framexbottommargin=4pt,
         %backgroundcolor=\color{lightgray},
         showstringspaces=false      % Leerzeichen in Strings anzeigen ?
 }
 \lstloadlanguages{
         %[Visual]Basic
         %Pascal
         %C,
         SQL,
         C++,
         XML,
         HTML
         %Java
 }
 
%  \lstset{
% 	language=[Visual]C++,
% 	keywordstyle=\bfseries\ttfamily\color[rgb]{0,0,1},
% 	identifierstyle=\ttfamily,
% 	commentstyle=\color[rgb]{0.133,0.545,0.133},
% 	stringstyle=\ttfamily\color[rgb]{0.627,0.126,0.941},
% 	showstringspaces=false,
% 	basicstyle=\small,
% 	numberstyle=\footnotesize,
% 	numbers=left,
% 	stepnumber=1,
% 	numbersep=10pt,
% 	tabsize=2,
% 	breaklines=true,
% 	prebreak = \raisebox{0ex}[0ex][0ex]{\ensuremath{\hookleftarrow}},
% 	breakatwhitespace=false,
% 	aboveskip={1.5\baselineskip},
%   columns=fixed,
%   upquote=true,
%   extendedchars=true,
% % frame=single,
% % backgroundcolor=\color{lbcolor},
% }
 
    %\DeclareCaptionFont{blue}{\color{blue}} 

  %\captionsetup[lstlisting]{singlelinecheck=false, labelfont={blue}, textfont={blue}}
  \usepackage{caption}
\DeclareCaptionFont{white}{\color{white}}
\DeclareCaptionFormat{listing}{\colorbox[cmyk]{0.43, 0.35, 0.35,0.01}{\parbox{\textwidth}{\hspace{15pt}#1#2#3}}}
\captionsetup[lstlisting]{format=listing,labelfont=white,textfont=white, singlelinecheck=false, margin=0pt, font={bf,footnotesize}}
\renewcommand*\lstlistingname{Broncode}

\usepackage{array}
\usepackage{tabularx}
\usepackage[latin1]{inputenc}
\usepackage[table]{xcolor}

\usepackage{placeins}

\newenvironment{packed_itemize}{
\begin{itemize}
  \setlength{\itemsep}{1pt}
  \setlength{\parskip}{0pt}
  \setlength{\parsep}{0pt}
}{\end{itemize}}

\begin{document}

\begin{preface}
	Allereerst wil ik mijn promotor Prof. dr. ir. Philip Dutr\'e en begeleider Dr. Benedict Brown bedanken om mij de kans te geven dit interessante project te doen. Dank ook aan Dr. Ares Lagae en Prof. dr. Danny De Schreye voor hun interesse in dit werk en de tijd te nemen die nodig is om het aandachtig te doorlezen.\\
	
	Het zou een doodzonde zijn om niet te benadrukken wat een goede begeleider Benedict is geweest, hij was er altijd als ik hem nodig had om mij te voorzien van nieuwe invalshoeken en manieren van werken. Zijn diepe kennis van het grotere project waarin deze thesis zich bevindt is van een onontbeerlijke waarde geweest.\\
	
	Dit project vergde veel inspanningen, niet alleen van mezelf, maar ook van anderen. Ik wil dan ook graag mijn ouders, vriendin, vrienden en kennissen bedanken voor hun niet aflatende steun en inzet. Ook de bijdrage van de K.U.Leuven kan niet onderschat worden. De meer dan bekwame docenten hebben mijn interesse in het vak enkel doen toenemen.\\
\end{preface}

\tableofcontents*

\begin{abstract}
	Het reconstrueren van archeologische fresco's is een domein waar computers nuttig werk kunnen verrichten, zowel voor archiveringsdoeleinden als voor digitale reconstructie. De algoritmen voor het automatisch ontdekken van passende fragmenten zijn reeds erg geavanceerd, evenwel moeten alle resulterende paarvoorstellen nog steeds verwerkt worden. Huidige tools zijn krachtig maar missen integratie, gebruiksvriendelijkheid en uitbreidbaarheid.
	
	Dit eindwerk tracht hierin verandering te brengen door het ontwikkelen van een platform dat doorgedreven analyse en manipulatie van voorstellen toelaat.

	Het resultaat is een applicatie die toelaat om verschillende personen vanop een afstand aan eenzelfde collectie te laten werken. Netwerktoegankelijke data zorgt ervoor dat de meest actuele informatie steeds beschikbaar is, terwijl de synchronisatie erover waakt dat er geen waardevolle vondsten verloren gaan.
	
	Tegenover het vroegere systeem is de sterk uitgebreide analysecapabiliteit ook nuttig, aangezien er hiermee complexe vragen over de collectie kunnen gesteld worden.
	
	Als laatste is het mogelijk om snel modules te maken die nieuwe visualisaties toelaten en deze eenvoudig aan te koppelen op het datasysteem. Enkele voorbeelden werden reeds gemaakt voor dit eindwerk.
\end{abstract}

%http://people.cs.kuleuven.be/~ronald.cools/MasterProef/titelbladFirW.html
% hoe dien je de thesis elektronisch in (via Toledo):
% https://eng.kuleuven.be/personeel/ethesis/ethesis-hdl-student.pdf

% Extra info: https://eng.kuleuven.be/onderwijs/studenten/masterproef/index.html

% Richtlijnen over wat er in je thesis MOET staan (abstract, \ldots): https://eng.kuleuven.be/studenten/masterproef/tips_nl.pdf
% Zie ook de thesis van Geert na: 

% Een lijst van figuren en tabellen is optioneel
%\listoffigures
%\listoftables

% Bij een beperkt aantal figuren en tabellen gebruik je liever het volgende:
\listoffiguresandtables

% \chapter{Lijst van termen}
% %\section*{Termen}
% \begin{flushleft}
%   %\renewcommand{\arraystretch}{1.1}
%   \begin{tabularx}{\textwidth}{l|r}
%   	RDBMS & Relational Database Management System / Relationeel databasebeheersysteem \\
%   	SQL & Structured Query Language / Gestructureerde querytaal, wordt ook gebruikt om te refereren naar databases die verzoeken in deze taal accepteren \\
%   	NoSQL & Gebruikt voor een klasse van databases die niet gebaseerd zijn op de relationele calculus en een zich concentreren op gedistribueerde data. De term is niet zo accuraat maar het is de meest populaire om deze databses onder te groeperen
%   \end{tabularx}
% \end{flushleft}

% Nu begint de eigenlijke tekst
\mainmatter

\chapter{Inleiding}
\label{inleiding}
Het reconstrueren van fresco's waarvan in opgravingen fragmenten gevonden worden is een moeilijke taak. Men kan het vergelijken met het oplossen van een enorme puzzel waarvan de stukken arbitraire vormen hebben, de meesten hun originele kleur zijn verloren en er vele anderen ontbreken. Daarbovenop ondervinden veel fragmenten erosie over de eeuwen heen, waardoor ze niet meer perfect op elkaar passen en confirmatie nog moeilijker wordt.\\

\begin{figure}[ht]
	\begin{center}
		\includegraphics[width=0.6\columnwidth]{images/WDC7_26.JPG}
		\caption{Een bak vol fragmenten die rond dezelfde locatie zijn gevonden, sommige zijn reeds aan elkaar gezet.}
		\label{fig:bakinleiding}
	\end{center}
\end{figure}

De stukken met een nog zichtbaar geometrisch patroon of van de rand van het fresco zijn in vergelijking met de anderen eenvoudig met elkaar te verbinden. Zij zijn door een ervaren archeoloog zonder hulp in elkaar te passen. De overige fragmenten die minder informatie bevatten zijn echter een nachtmerrie om aan elkaar te puzzelen. Er ontbreekt informatie die een structuur vanop een afstand laat zien, het menselijke visuele systeem is blijkbaar niet erg geschikt enkel grillige randen te vergelijken en aan elkaar te zetten. Het enige alternatief lijkt om lokaler te zoeken en elk fragment te vergelijken met elk ander fragment. Deze aanpak is natuurlijk niet mogelijk, er zijn miljoenen combinaties van fragmenten mogelijk.\\

In deze context situeert zich het thera\footnote{Thera is de oude naam voor het huidige griekse eiland genaamd Santorini, waar het project voor het eerst in de praktijk werd toegepast.} project, dat probeert om het werk van de archeoloog gemakkelijker te maken door middel van een software platform~\cite{Brown2008}. De
redenering achter het project is dat een computer de ondankbare taak voorgesteld in de vorige paragraaf kan automatiseren. Dergelijk systeem werd in 2007 aan de \emph{Princeton} universiteit in Amerika geconcipieerd. Sindsdien is er tot op de dag van vandaag door verschillende onderzoekers van over de hele wereld aan gewerkt. De werking van het systeem wordt in het volgende hoofdstuk nader toegelicht.\\

Deze thesis draait rond het maken van een uitbreiding op het platform. De uitbreiding moet de gebruikers van het systeem in staat stellen om de beschikbare data op nieuwe manieren te gebruiken, te visualiseren, aan te passen en te delen met medeonderzoekers. Aangaande terminologie zullen een paar woorden veelvuldig terugkomen: de termen fragment, brokstuk of gewoonweg stuk verwijzen altijd naar een enkel gebroken deel van het originele fresco. De termen fragmentpaar, paar en voorstel zijn gereserveerd voor een aaneenkoppeling van twee fragmenten op een bepaalde plaats. Het valt te benadrukken dat twee fragmenten met elkaar verschillende paren kunnen vormen, indien zij op verschillende punten raken (ook al liggen die punten niet zo ver uit elkeaar). Om deze reden wordt een paar ook vaak een voorstel genoemd, om te benadrukken dat het een mogelijke configuratie is, maar geen zekere. Verder verwijst deze tekst meermaals naar (automatische) herkenners, dit zijn de identificatiealgoritmen die twee fragmenten aan elkaar proberen passen.\\

\section{Overzicht}
Hoofdstuk~\ref{hoofdstuk:overzicht} geeft een overzicht van het bestaande werk en hoofdstuk~\ref{hoofdstuk:doelen} geeft aan op welke vlakken het project zal uitgebreid worden en waarom. Vervolgens behandelt hoofdstuk~\ref{hoofdstuk:ontwerp} het algemene ontwerp van de applicatie. Hoofdstukken~\ref{hoofdstuk:database},~\ref{hoofdstuk:synchronisatie} en~\ref{hoofdstuk:modules} gaan dieper in op enkele specifieke implementaties van het project zoals de database, de synchronisatie en de modules. Tenslotte volgt een besluit in hoofdstuk~\ref{hoofdstuk:besluit}.

\chapter{Overzicht van het thera project}
\label{hoofdstuk:overzicht}
Zoals eerder vermeld bouwt deze thesis verder op een reeds bestaand project. Om ze beter te kunnen situeren in het geheel is het handig om eerst even het thera project te bekijken en zodoende te verduidelijken waar het thesisproject in past.

\section{De opdelingen van het thera project}
Ruwweg gezien kan het reconstrueren van een fresco met behulp van de computer opgedeeld worden in 3~fasen.

\begin{description}
	\item[Acquisitie] Alle gevonden fragmenten worden ingescand met behulp van 3D en 2D-scanapparatuur om zo een virtueel model van elk stuk te bouwen, zie \cite{Brown2008}.  
	\item[Identificatie] Vervolgens worden deze virtuele fragmenten aan een zogenaamde \emph{matcher} (automatische herkenner) gegeven, die voor elk fragmentenpaar gaat kijken of ze mogelijk op elkaar passen. Er zijn verschillende types ontwikkeld. E\'en ervan (genaamd \emph{RibbonMatcher}~\cite{Brown2008}) kijkt enkel naar de randen van de brokstukken om te zien of ze in elkaar sluiten. Hierdoor is het in staat om passende fragmenten te ontdekken die voor mensen moeilijk te vinden zijn wegens te weinig distinctieve attributen zoals tekeningen. Het onderzoek gaat verder, in 2010 werd er een nieuw type ontwikkeld dat zijn analyse baseert op een combinatie van verschillende eigenschappen, zoals de sporen die een borstel kan nalaten bij het kleuren van een fresco of zelfs de beoordeling van de eerder genoemde \emph{RibbonMatcher}~\cite{TolerFranklin2010}.
	\item[Classificatie \& Reconstructie] De derde en laatste stap bestaat uit het classificeren van wat de vorige stappen produceren. Elk voorgesteld paar moet gecontroleerd worden op validiteit. Verschillende statussen kunnen toegekend worden aan een paar, zoals: \emph{geconfirmeerd}, \emph{misschien}, \emph{nee}, \emph{in conflict met een ander paar}, et cetera. Het algoritme produceert in de regel zeer veel paren van brokstukken, waarvan slechts een klein deel correct kan zijn. De drempel voor het beslissen van wat een paar kan zijn en wat niet wordt in de vorige stap zo laag ingesteld omdat men wil vermijden dat twee fragmenten die toch passen genegeerd worden. Anders gezegd, de kost van ``\emph{false negatives}'' wordt hoog ingeschat. Uiteindelijk zal hieruit een gereconstrueerde fresco moeten ontstaan.
\end{description}

\section{Reconstructie, de bestaande oplossingen}
Het thesisproject besproken in deze tekst tracht de reeds bestaande hulpmiddelen van de reconstructiefase aan te vullen. Hievoor werden reeds programma's ontwikkeld, namelijk \emph{Griphos} en \emph{Browsematches}. Hierop volgt een bondige bespreking van beide programma's, wat ze kunnen en niet kunnen, en eventuele voor- en nadelen. Deze factoren hebben een belangrijke rol gespeeld bij het bepalen van de richting van deze thesis.

\subsection{Griphos}

Griphos was de eerste applicatie gericht op het weergeven en beoordelen van de resultaten van de voorgaande stappen. Het werd ontworpen met het doel een centraal zenuwstelsel te zijn voor alle informatie en met deze uiteindelijk een (zo goed mogelijk) gereconstrueerd fresco te maken. Het is mogelijk om zowel aparte fragmenten als automatisch herkende fragmentparen op een virtueel tafelblad te plaatsen en te manipuleren (zie figuur \ref{fig:griphostafelblad}). De idee achter een dergelijke voorstelling komt voort uit een rechtstreekse vertaling naar de computer van wat een archeoloog op een werkelijk tafelblad doet.\\

\begin{figure}[ht]
	\begin{center}
		\includegraphics[width=.8\columnwidth]{images/griphos-01-cut.png}
		\caption{Een voorbeeld Griphos tafelblad, 1 voorgesteld paar en 2 aparte fragmenten zijn reeds ingeladen}
		\label{fig:griphostafelblad}
	\end{center}
\end{figure}

Griphos wordt echter geplaagd door een paar problemen: het is traag en biedt geen goede manier om de talloze gegenereerde paarvoorstellen na te kijken. De gebruikte metafoor van het virtuele tafelblad waarin gepuzzeld kan worden stelt een belangrijk deel van het reconstructieproces voor, maar schiet tekort als men de resultaten van automatische fragmentpaarontdekking op vloeiende wijze in het proces wil betrekken. Het probleem lijkt te zijn dat er te veel informatie is en Griphos geen goede manier biedt om door de bomen het bos te zien. De aanzienlijke traagheid van sommige delen van het programma komen vooral voort uit de manier van dataopslag voor paren (XML bestanden) en de complexiteit van de visualisatie (afbeeldingen van hoge kwaliteit en/of gedetailleerde 3D-modellen). Dit zorgt soms voor een onaangename werkervaring, zelfs indien men er toch in slaagt de juiste fragmentparen te lokaliseren. Desalniettemin is het een krachtig programma dat veel functionaliteit biedt voor de detailinspectie van brokstukken.\\

Het heeft zijn nut al op verschillende vlakken bewezen, behalve detailinspectie is het bijvoorbeeld ook in staat om de posities van fragmenten in een bak te onthouden. Dit betekent een grote snelheidswinst wanneer men bijvoorbeeld denkt een goed paar te hebben gevonden en men wil dit met echte fragmenten verifi\"eren. Als beide fragmenten in dezelfde bak liggen kan het correcte tafelblad ingeladen worden, Griphos kan vervolgens de gezochte fragmenten laten oplichten. Dit is veel effici\"enter dan fragmenten opsporen door vormen te vergelijken, zeker omdat de brokstukken vaak moeilijk te onderscheiden zijn zoals te zien valt in figuur \ref{fig:griphosbak}. 

\begin{figure}[ht]
	\begin{center}
		\includegraphics[width=.8\columnwidth]{images/griphos-bak-01.png}
		\caption{Griphos kan gebruikt worden om de posities van fragmenten in hun opslagplaats te onthouden en vervolgens snel terug te vinden. Het kan ook een foto van de bak waarin ze liggen onder het virtuele tafelblad projecteren. (De brokstukken die buiten de bak staan weergegeven zijn verplaatst geweest naar een andere bak)}
		\label{fig:griphosbak}
	\end{center}
\end{figure}



\subsection{Browsematches}

Een eerste prototype om de in Griphos ontbrekende delen aan te vullen, werd Browsematches genoemd. Het gebruikt eerst de visualisatiecapabilitieiten van Griphos om kleine afbeeldingen te nemen van elk bestaand paar met aan de linkerkant de doorsnede van hun raakvlak. Vervolgens toont het een scherm vol paren en kan men met de pijltoetsen navigeren tussen schermen. \\

\begin{figure}[ht]
	\begin{center}
		\includegraphics[width=.8\columnwidth]{images/browsematches-01-cut.png}
		\caption{Browsematches in werking, de balken boven de paren duiden een validatie door de gebruiker aan. Rood betekent bijvoorbeeld ``dit voorstel is zeker niet juist''}
	\end{center}
\end{figure}

Dit kleine programma groeide uit noodzaak: het valideren van paarvoorstellen is zo omslachtig met Griphos dat Browsematches in korte tijd werd gemaakt om het proces te stroomlijnen. Bij het begin van de thesis, om kennis te maken met het thera project, werd een verbinding gemaakt met Griphos zodat voorstellen die interessant waren in detail bestudeerd konden worden.\\

Browsematches erfde echter ook enkele van de nadelen van Griphos. Het inladen van de data is traag en vergt zelfs voor kleine datasets veel geheugen. Daarbij is het moeilijk om de informatie uit te breiden of deze gemakkelijk te combineren met de bevindingen van andere onderzoekers. Daarenboven is Browsematches een prototype en niet bedoeld voor algemeen gebruik. Het simpele maar succesvolle concept diende als inspiratie en basis voor deze thesis. 

%%% Local Variables: 
%%% mode: latex
%%% TeX-master: "masterproef"
%%% End: 

\chapter{Doelen \& Motivatie}
\label{hoofdstuk:doelen}

Het hoofddoel van het thera project is het reconstrueren van fresco's uit de oudheid zo gemakkelijk mogelijk te maken. Er moet een manier gevonden worden om een waardevolle contributie te maken aan het thera-ecosysteem zodat het zijn doel beter kan vervullen. Dit is onder andere mogelijk door delen die ontbreken aan de vorige oplossingen aan te maken of bestaande componenten veelzijdiger te maken.\\

Gezien de huidige stand van zaken besproken in hoofstuk \ref{hoofdstuk:overzicht}, valt het op dat er nog behoorlijk wat dingen kunnen toegevoegd worden op het gebied van informatie en het visualiseren ervan. De ongelooflijke hoeveelheid data die het thera project reeds heeft geproduceerd en blijft produceren kan op een betere manier behandeld worden zodat het ware potentieel ervan naar de oppervlakte komt. De laatste stap in het sterk geautomatiseerde virtuele reconstructieproces wordt gekenmerkt door een nood aan interactie met de mens die elk soort hulpmiddel moet krijgen om zo correct en snel mogelijke beslissingen te maken. Natuurlijk gaat er niets boven de feitelijke fragmenten fysisch vastnemen en ze aan elkaar te proberen zetten. Maar gezien dit een zeer tijdsrovende bezigheid is, moet dit zo veel mogelijk beperkt worden.\\

Er werd gesproken over Griphos en diens weinig ontwikkelde ondersteuning voor het behandelen van automatisch voorgestelde fragmentparen, alsook over de eerste poging om dit recht te trekken: Browsematches. Deze thesis tracht de lijn van Browsematches verder te zetten, vertrekkende van de goede idee\"en en ervaring waaraan het zijn ontstaan te danken heeft. Dit houdt in dat de focus niet meer zozeer ligt op het actief puzzelen en brokstukken op een tafelblad plaatsen, maar eerder op het beoordelen van de omvangrijke verzameling voorstellen, waarvan slechts een zeer klein deel correct kan zijn. De hoop en verwachting is dat dit zeer kleine deel meteen ook een significante hoeveelheid van de werkelijk overblijvende paren voorstelt.\\

Daarmee valt te benadrukken dat dit project geen vervanging probeert te zijn van de bestaande software, het is eerder bedoeld als een complement. Merk op dat gevalideerde paren niet kunnen conflicteren en dus rechtstreeks in een groep op een tafelblad geplaatst kunnen worden. In de limiet, wanneer alle mogelijke paren correct geclassificeerd zijn vloeit hieruit op natuurlijk wijze een (zo compleet mogelijk) fresco voort. Daarom valt de implementatie in de thesis te zien als een extra tussenstap, chronologisch v\'o\'or het plaatsen van de fragmenten op een tafelblad en na het uitvoeren van de herkenningsalgoritmen. Figuur \ref{fig:flow} stelt dit visueel voor.

\begin{figure}[ht]
	\begin{center}
		\includegraphics[width=1.0\columnwidth]{images/flowchart-focus-01.png}
		\caption{Het huidige proces met de aanvullingen van de thesis in het blauw}
		\label{fig:flow}
	\end{center}
\end{figure}

Een van de meest fundamentele verschillen tussen de centrale filosofie van Griphos en die van dit project is dus dat terwijl bij Griphos de focus ligt op aparte geplaatste fragmenten, het thesisproject eerder de automatisch gevonden paren behandelt. Het is in Griphos wel degelijk mogelijk om voorstellen van de automatische paarherkenning in te laden maar de naald in de hooiberg vinden is moeilijk.\\

\section{Aspecten waarop de thesis tracht te verbeteren}
Hieronder staan verschillende deelaspecten waarop dit thesisproject verbeteringen tracht te maken. Bij elk aspect staat beschreven wat de vereisten zijn waar het project aan zal moeten voldoen en eventueel een voorbeeldscenario. Men kan deze aspecten desgewenst onafhankelijk bekijken maar vaak steunen ze op elkaar om werkelijk tot hun recht te komen. Een visualisatiemethode heeft bijvoorbeeld een manier nodig om de data die het wil visualiseren te krijgen. Omgekeerd is een databeheersysteem niet bijzonder nuttig zonder de mogelijk de data te manipuleren. Om de verbeteringen en idee\"en te bundelen en te testen is er ook een applicatie gemaakt die dient om reeds een voorproef te geven van wat er mogelijk is met de ontwikkelde technologie.  

\subsection{Integratie}
Zoals eerder vermeld staat dit thesisproject niet alleen maar maakt het deel uit van een groter geheel. Binnen de grenzen van het mogelijke zou er moeten rekening gehouden worden met de integratie van de geschreven code met die van de andere onderzoekers. Dit verhoogt de kans dat het werk aanvaard wordt en ingang vindt in andere subprojecten. 

\subsection{Collaboratie}
Ideaal gezien zouden archeologen steeds toegang moeten krijgen tot hun project op eender welk moment vanop eender welke plaats. Dit kan een gedeelde collectie zijn die over het internet beschikbaar wordt gesteld, of een lokale kopie. Een mogelijk scenario hierbij is dat een ervaren archeoloog in de Verenigde Staten gevraagd wordt om zijn opinie te geven over de huidige stand van zaken (reeds ge\"identificeerde correcte paren, moeilijke gevallen, \ldots). Alle aanpassingen en commentaren die hij maakt worden automatisch ingevoegd en centraal beschikbaar gesteld voor de onderzoekers ter plekke. Op termijn moet het bijvoorbeeld zelfs mogelijk worden om amateurs te laten kijken naar de voorstellen en hun beoordeling te gebruiken om de nog na te kijken voorstellen te rangschikken.\\

Van cruciaal belang is dat alle verzamelde data (zoals de classificatie van voorstellen) op een robuuste manier opgeslagen, gedeeld en ge\"incorporeerd kan worden. De toegang naar en manipulatie van deze informatie moet effici\"ent zijn. De huidige oplossingen zijn hiervoor ontoereikend en traag, zoals besproken in hoofdstuk \ref{hoofdstuk:overzicht}. De mogelijkheid van meerdere en zelfs lokale kopie\"en impliceert ook de nood om te kunnen synchroniseren\footnote{Naar het model van de zogenaamde \emph{Distributed Version Control System (DCVS)} systemen zoals Git, Mercurial, \ldots}. Dit is ook nodig omdat er reeds verschillende huidige verzamelingen bestaan die eventueel in het nieuwe systeem ge\"importeerd en gecombineerd moeten worden. Dergelijk systeem is ook nuttig voor het maken van (kleinere) pocketversies en in gebieden waar de internetconnectiviteit niet adequaat of onbestaande is. Belangrijk is dat er steeds een manier is om waardevolle data te combineren en aan te vullen zodat niets verloren gaat.

\subsection{Schalering}
De bestaande oplossingen voor het valideren van voorgestelde paren werken noodgedwongen met een sterk gereduceerde verzameling. E\'en van de redenen hiervoor is het gebruik van een groot XML bestand om alles in op te slaan. Deze techniek werkt goed voor bijvoorbeeld het opslaan van tafelbladen --- waar er bijvoorbeeld 50 fragmenten en hun locaties moeten opgeslagen worden --- maar schiet tekort voor paarvoorstellen. Een snelle rekensom geeft de reden aan: een laag aantal fragmenten voor een specifieke opgraving is bijvoorbeeld 1500. De meeste paarherkenners gaan alle brokstukken een keer met elk andere brokstuk vergelijken en zodoende het meest waarschijnlijke aankoppelingspunt vinden. Het is zelfs mogelijk dat een fragment meerdere keren een plausibel paar vormt met eenzelfde stuk. Naar onder afgerond komen uit deze stap reeds 2 miljoen configuraties gerold, de ene wat waarschijnlijker dan de andere. Dit nummer stijgt kwadratisch in het aantal fragmenten en zelfs met hogere drempels om volgens een bepaalde maatstaf paren automatisch te verwerpen stijgt het aantal configuraties snel.

\subsection{Gebruiksvriendelijkheid}
De uiteindelijke gebruikers van de applicatie zijn niet de ontwikkelaars van het thera project zelf, maar de archeologen. Om deze reden is het belangrijk dat er rekening gehouden wordt met de noodzaak van een visueel aangename en intuiti\"eve gebruikservaring. Om deze intu\"itiviteit te bereiken moet in het programma steeds de aandacht gevestigd blijven op hetgeen het belangrijkst en meest herkenbaar is: de paren en hun fragmenten. Bij voorkeur moet elke operatie gemakkelijk ontdekbaar zijn in de context waar ze kan gebruikt worden, met eventueel een woordje uitleg erbij.\\

Volgens vele studies op het gebied van gebruikersinterfaces \cite{Hoxmeier00,Shneiderman84,Nielsen94} geraken gebruikers gefrustreerd vanaf een operatie een zekere tijd duurt. Deze frustratiedrempel hangt een af van de aard van de operatie (het resultaat), de frequentie waarmee die uitgevoerd wordt, of er visuele tekenen van voortgang zijn en of er tijdens het wachten (steeds) iets anders kan uitgevoerd worden. Om deze reden is het belangrijk dit aspect in acht te nemen bij het ontwikkelen van de applicatie. Dit is vooral zo omdat veel van de acties die mogelijk moeten zijn het potentieel hebben traag te lopen en dus de gebruiker te frustreren. Een algemene vaststelling: de tijd die een operatie mag innemen is omgekeerd evenredig met de frequentie waarmee deze operatie moet uitgevoerd worden. Deze regel in acht nemend is het duidelijk dat bijvoorbeeld het inladen van een scherm vol voorstellen zoals bij Browsematches, het veranderen van een attribuut van een paar, het filteren en sorteren en dergelijke meer acties zijn die met de grootst mogelijke snelheid moeten worden uitgevoerd. Hoe functioneel ook, een programma dat sloom reageert en elke computer op z'n knie\"en dwingt zal zo weinig mogelijk gebruikt worden \cite{Joel2001}.\\

// hoort mss bij [DESIGN]
De gebruikersinterface van de oude programma's was goed en kon effici\"ent gebruikt worden mits enige training. Maar voor het ondersteunen van collaboratief werken moeten er natuurlijk allerhande nieuwe operaties toegevoegd
worden. Tijdens het implementatieproces werd het duidelijk dat de reeds bestaande code van het Browsematches niet uitbreidbaar genoeg was en die van Griphos te complex en belangrijk. Daardoor werd de beslissing genomen om de
basisinterface van Browsematches over te nemen maar alle onderliggende code te herschrijven zodat die uitbreidbaar zou zijn. Bij het opnieuw construeren van dit alles zijn er een aantal verbeteringen gebeurd die niet meteen te maken hebben met het collaboratie aspect maar wel met de workflow van het classificeren. Dit werd gedaan om het hele programma gebruiksvriendelijker en krachtiger te maken in functie van het hoofddoel van het project.

\subsection{Uitbreidbaarheid}
Het samenstellen van werken uit de oudheid is een zeer vakkennis- en ervaringsintensief proces. Hoewel men luistert naar wat de archeologen hierover te vertellen hebben --- wat ze graag zouden zien of kunnen doen --- zijn er vele zaken die nu nog niet duidelijk zijn maar in de toekomst zeker aan het licht zullen komen. Dit kan bijvoorbeeld zijn omdat de onderzoekers in kwestie niet goed kunnen uitleggen waar ze naar kijken of hoe ze zoeken: na zovele jaren vertrouwen ze op hun moeilijk te defini\"eren intu\"itie. Anderzijds is het vertalen van het proces om fresco's samen te stellen naar de computer nog niet zo vaak geprobeert in het verleden. Dit betekent dat een goede werkwijze in de realiteit misschien niet zonder meer de effici\"entste is als men de transitie naar virtueel reconstrueren maakt (zoals bij Griphos). Daarbovenop zijn er nog vele kansen om innovatieve nieuwe technieken aan te wenden die niet werkbaar zijn als men enkel over fysische fragmenten beschikt.\\

Het is dus onwaarschijnlijk dat het laatste woord over de ideale metafoor reeds gezegd is waardoor er een grote kans is dat het platform zal moeten vervangen of herbouwd worden als het niet met uitbreidbaarheid in gedachte ontworpen wordt. Er moeten moeiteloos nieuwe delen aan de applicatie en de onderliggende lagen kunnen toegevoegd worden om snel nieuwe idee\"en te incorporeren. Dit alles moet best mogelijk zijn zonder de ervaring van gebruikers met oudere versies of andere gebruikersinterfaces te degraderen.

\subsubsection{Data}
De eerder besproken uitbreidbaarheid die nodig is manifesteert zich op het niveau van de data bijvoorbeeld bijboorbeeld op deze manier: een onderzoeker vindt een nieuw algoritme om fragmentparen te rangschikken op ``goedheid'' of men wil informatie over het dikteverschil tussen twee fragmenten opslaan. Idealiter zouden deze zaken als een attribuut bij een paar moeten kunnen toegevoegd worden zodat elke gebruiker er op kan zoeken en sorteren zonder iets extra te hoeven doen.\\
 
Enkel data die op geen enkele manier om te vormen valt naar een attribuut van een enkel paar zal een speciale module vereisen om te kunnen gebruiken. Een vereiste is natuurlijk wel dat dit geen effect mag hebben op de delen van het platform die hier geen weet van hebben.
 
\subsubsection{Visualisatie}
Er zijn vele visualisatiemanieren denkbaar die elkaar kunnen aanvullen bij het volbrengen van het zoek- en classificeerprocess. Zo stelt men zich bij de uitdrukking ``fresco's samenstellen'' waarschijnlijk een grote puzzel voor waar men ten alle tijde het overzicht kan behouden en stukken proberen te passen.\footnote{Het Griphos programma gaat uit van het idee van kleine beheersbare stukken van de reuzenpuzzel (elk tafelblad zou een verzameling kunnen zijn van stukken die gerelateerd waren, bijvoorbeeld door hun vindplaats)} Deze puzzel visualisatie is visueel aantrekkelijk en biedt het menselijke patroonherkenningsvermogen\footnote{Iets waar computers de mens nog altijd niet in evenaren. Een ander thera subproject onderzoekt wel de mogelijkheid om zogenaamde 'clusters' te identificeren en te gebruiken voor reconstructie.} de mogelijkheid om zich van zijn beste kant te laten zien. Echter, door de grote hoeveelheid aan fragmenten en dus mogelijke paren is het moeilijk om hier aan te beginnnen. Maar, hoe meer reeds geconfirmeerde paren er zijn, hoe duidelijker het globale beeld kan worden. Dit staat toe om een overzicht te krijgen van de vooruitgang en gerichter te zoeken naar stukken die nog ontbreken. Dit is een voorbeeld van een macro-perspectief.\\

Een alternatief is te beginnen door te kijken naar waar de computer w\'el goed in is: fragmenten aan elkaar passen en rangschikken naar kans/overeenkomst/et cetera. Door een gemakkelijk navigeerbare lijst op te stellen van alle voorstellen die de reconstructie algoritmes hebben gedaan, kunnen er snel op elkaar passende fragmenten ge\"identificeerd worden. Dit kan men zien als een micro-perspectief of \emph{bottom-up} manier om fresco's te reconstrueren. Het is ook de aanpak die in Browsematches gebruikt wordt. Nadat men bijvoorbeeld met deze manier een deel acceptabele paren heeft ge\'identificeerd, kunnen deze bijvoorbeeld weergegeven worder als een grote puzzel en dienen zij als beginpunt om verder te puzzelen. Dit maakt het globale beeld veel informatiever: stel dat duidelijke "gaten" ontstaan in een resem goede fragmentparen, dan kan er gericht gezocht worden naar een fragment dat erin past door te zoeken naar een fragment dat met elk van deze insluiters past (indien het gevonden werd bij de opgraving).\\

Kortom, het is duidelijk dat een visualisatie die in alle gevallen de meest geschikte is niet bestaat. De beschikbare informatie moet soms gewoon op andere manieren worden weergegeven. Om deze reden is het wenselijk om het mogelijk te maken snel nieuwe visualisaties in te bouwen die kunnen communiceren met andere delen van de applicatie en eventueel de informatie manipuleren. 
\chapter{Ontwerp van het project}
\label{hoofdstuk:ontwerp}

Eenmaal de belangrijkste vereisten gekend zijn kan er een ontwerp opgetekend worden. Daarom even de grote lijnen die te concluderen vallen uit hoofdstuk \ref{hoofdstuk:doelen}:

\begin{itemize}
  \item De data en het visuele aspect van de applicatie moeten zo ontkoppeld mogelijk zijn, elk moet apart uitbreidbaar zijn
  \item Zeer grote hoeveelheden data moeten vlot kunnen behandeld en genavigeerd worden: zoeken, filteren, sorteren, aanpassen, \ldots
  \item De data moet zowel centraal als decentraal toegankelijk zijn en synchronisatie toestaan
  \item Compatibiliteit met de reeds geschreven onderdelen van het project moet indien mogelijk bewaard blijven
  \item Om het ontwerp te verifi\"eren moet een werkend programma gemaakt worden, gebruiksvriendelijkheid, functionaliteit en snelheid zijn hierbij belangrijk
\end{itemize}

\section{De grote lijnen}
Van alle vereisten die rechtstreeks invloed kunnen uitoefenen op de architectuur, is de splitsing van de data en de visualisatie waarschijnlijk de meest fundamentele. Een beproefde aanpak om dit te realizeren is het ontwerppatroon genaamd \emph{Model-View-Controller (MVC)}\footnote{Een algemeen overzicht kan men bijvoorbeeld bekomen op \url{http://en.wikipedia.org/wiki/Model-view-controller}}.\\

Het doel is een ontwerp te maken waarin we een visualisatiemethode kunnen verbinden aan de juiste databron. Enkele zaken moeten echter nog vastgelegd worden, namelijk: wat is de basiseenheid van data? Waar zal een visualisatiemodule achter vragen? Zoals in figuur~\ref{fig:flow} te zien is, ligt de focus bij dit project op koppelingen van fragmenten in plaats van fragmenten op zich. Er zijn op het eerste zicht twee alternatieven om dit te modelleren. Een eerste mogelijkheid is een soort van \emph{MatchedFragment} die een fragment beschrijft plus een lijst met alle fragmenten die er potentieel aan gekoppeld kunnen worden en op welke locatie. De tweede mogelijkheid is om elk paar een apart object te laten voorstellen (bvb. genaamd \emph{FragmentPair}). Het eerste alternatief lijkt het programmatische voordeel te hebben dat het gemakkelijk is om na te kijken of een fragment reeds ``bezet'' is. Ook zou het dan mogelijk zijn om bijvoorbeeld een grafe op te stellen door van brokstuk naar brokstuk te springen. Echter, dit soort opstelling bevat veel redundantie, een fragment zal op die manier een verwijzing met attributen naar een fragment bevatten, en dit fragment zal op zijn beurt een identieke omgekeerde verbinding hebben. De redundantie vermijden en een verwijzing als een apart object voorstellen waar beide fragmenten naar kunnen verwijzen is eigenlijk niets anders dan de tweede optie (een \emph{FragmentPair}). Daarbij kan het probleem van hoe de ``bezetting'' van een object te weten te komen (alsook de grafe, zoals later zal aangetoond worden) opgelost worden door de vereiste zoekfunctionaliteit van het datamodel te benutten.\\

Eenmaal deze basiseenheid van informatie gekozen, valt de kern van de applicatie volgens MVC uit te beelden als in figuur~\ref{fig:basicprogramflow}.

\begin{figure}[ht]
	\begin{center}
		\includegraphics[width=1.0\columnwidth]{images/BasicExecutionFlow.png}
		\caption{Het abstracte model van de applicatie, links staat de \emph{View/Controller} en rechts het \emph{Model}. De controller stuurt een verzoek naar het model voor een bepaalde (sub)set van de data --- al dan niet gesorteerd --- en het model antwoord met alle paren die voldoen aan de criteria}
		\label{fig:basicprogramflow}
	\end{center}
\end{figure}

\section{Modulariteit}
Om het geheel uitbreidbaar te maken is een pluginsysteem gemaakt. Er is een hoofdapplicatie (codenaam ``Tangerine'') die de connectie maakt met de databeheerlaag en verschillende visualisatieplugins kan laden. Het basissysteem zonder modules bestaat uit een manier om een fragmenten en paren-database in te laden en te kiezen welke module op te starten.\\

De eerste en meest uitgewerkte daarvan werd \emph{MatchTileView} gedoopt. Het geeft de paren op dezelfde manier weer als het Browsematches prototype, maar is natuurlijk uitgebreid qua mogelijkheiden. De bespreking van de nieuwe functionaliteiten komt aan het bod in het hoofdstuk over modules.\\

\begin{figure}[ht]
	\begin{center}
		\includegraphics[width=1.0\columnwidth]{images/browsematches-to-tangerine-01.png}
		\caption{De manier van weergeven uit Browsematches werd gekopi\"eerd naar het nieuwe platform, met uitbreidingen}
		\label{fig:browsematchestotang}
	\end{center}
\end{figure}

Elke module krijgt van de applicatie een model toegewezen, dit kan een blanco model zijn zonder criteria of een gedeeld model. Een gedeeld model betekent bijvoorbeeld dat als er een een module beslist om te sorteren op een attribuut zoals ``het verschil van de dikte tussen twee fragmenten'', alle modules die gebruikmaken van ditzelfde model opeens over een gesorteerde dataset beschikken. Via speciale signalen worden zij hiervan op de hoogte gebracht, zodat ze kunnen beslissen of het nodig is een actie te ondernemen. Dit kan handig zijn voor pure visualizatieplugins die geen zoekmogelijkheden aan de gebruiker blootstellen, het kan dan vertrouwen op andere modules om data aan te leveren.\\

Indirecte communicatie via het model is (voorlopig) de enige manier waarop modules elkaar kunnen be\"invloeden. De structuur van de componenten ziet er uit als in figuur~\ref{fig:visualizationlayer}. Merk op dat er een plugin is (\emph{DetailView}) die geen gebruik maakt van fragmentparen maar eerder van een virtueel tafelblad net als Griphos. Zoals eerder aangehaald kunnen fragmentparen automatisch op een tafelblad gezet worden. Dit tafelblad kan dan in 3D weergegeven worden door \emph{DetailView}, waarover later meer.

\begin{figure}[h]
	\begin{center}
		\includegraphics[width=1.0\columnwidth]{images/VisualizationExtract.png}
		\caption{Een vereenvoudigde kijk op de componenten van de visualisatielaag, het hoofdscherm en het model. De componenten in het wit behoren tot de rest van het thera project en zijn niet gemaakt als deel van dit thesisproject.}
		\label{fig:visualizationlayer}
	\end{center}
\end{figure}

\section{De eenheid van informatie: het fragmentenpaar}

Speciale eigenschappen, write-through, \ldots

% DATA
\section{Het beheer van de data}

Redenering: Alles vloeit voort uit de paren, het is voorspelbaar (simpel voor te stellen) en ondubbelzinnig

Het zelf kunnen maken van paren is van secundair belang (volgens de thesis), zij kunnen door de HumanMatcher ingevoerd worden in de grote database

Griphos is zeer nuttig -> identificatie locatie van fragmenten in de bak, etc. -> Tangerine is het ontbrekende middendeel!

Dit betekent echter niet dat beide perspectieven elkaar uitsluiten, integendeeel.
 
[TODO: verhuizen naar ontwerp van database]

\subsection{Het oude systeem: XML-bestanden}

E\'en van de belangrijkste stappen op weg naar een collaboratieve applicatie, is het omvormen van\\

Het ontwerp van het database model en de implementatie details zijn groot genoeg om hun eigen hoofdstuk te verdienen, zij worden later in hoofdstuk [...] besproken.\\

\subsection{Ontleding van de data}
...relationeel van aard => SQL database

\subsubsection{Paren}

\subsubsection{Attributen van paren}

\subsubsection{Complexe informatie die niet in een simpel attribuut past}

\subsection{De vereisten van een database ontwerp}
Zoals in sectie [doelen] omschreven moet de applicatie in staat zijn om data op een uitbreidbare manier uit te lezen en te veranderen, liefst zonder 

\subsection{Complexe informatie voorstellen als een attribuut}
Meta-attributen, VIEWs

\subsection{Verschillende iteraties}

\subsection{Databases samenstellen}

\subsubsection{Geschiedenis bijhouden}

% VISUALISATIE
\section{Visualisatie, een manier om met de data te werken}

\subsection{Model-View-Controller}

\subsection{UML diagramma}

[maak UML diagramma]\\

\subsection{GUI}
Met behulp van de Qt toolkit [link]...

Multi-threading voor responsiviteit

\section{Integratie in thera project}
Gebruik van bibliotheken, ...
Extensie door refactoring: FragmentConf -> IFragmentConf ==> SQLFragmentConf | FragmentConf
Hierdoor is het nodig om de reeds bestaande code van het thera project om te zetten naar het gebruik van IFragmentConf waar mogelijk maar het aanmaken van FragmentConf anderzijds
of FragmentConf ==> SQLFragmentConf
Alternatieve oplossing, geen veranderingen in thera code maar SQLFragmentConf sleept dan veel onnodige ballast mee van FragmentConf

\section{Uitbreidbaarheid}
[afbeeldingen invoegen]\\


[afbeelding invoegen, selectie naar grafe transit!]\\
\chapter{Bespreking van het database ontwerp}
\label{hoofdstuk:database}
De visuele delen van het platform bekommeren zich met de \textbf{de manier waarop} er data moet weergegeven worden, de gebruiker kiest \textbf{wat} er moet weergegeven worden. Om de data hiervoor aan te voeren is er een datalaag nodig die elk verzoek juist en effici\"ent kan verzorgen.

\section{Vereisten}
De hoofdpunten waaraan deze laag moet voldoen werden reeds opgesomd in hoofdstuk~\ref{hoofdstuk:doelen}, namelijk: snelheid, universele toegang, synchroniseerbaarheid en maximale flexibiliteit voor data-analyse. Eerst wordt een analyse gemaakt van de structuur van de data. Vervolgens wordt er gekeken naar de huidige methode die het thera project hiervoor hanteert. Enkele alternatieven worden besproken en op basis daarvan wordt een keuze gemaakt.

\subsection{Ontleding van de data}
De kern van de data is een verzameling fragmentparen, deze bezitten elk een paar kernattributen en vele optionele. De kernattributen beperken zich tot de namen (of een identificatienummer) van de fragmenten, en een transformatiematrix die het \'ene brokstuk aan het andere koppelt, er is geen enkel paar dat deze gegevens mist. Verder kunnen de herkenningsalgoritmen en eventueel post-processingalgoritmen allerlei soorten data toevoegen als attributen. Dit kan gaan om een maatstaf afhankelijk van het pasproces zoals de zogenaamde \emph{RibbonMatcherError}\cite{Brown2008} of een algemene eigenschap zoals het overlappende volume, in hoofdzaak gaat het hier om decimale en re\"ele getallen. Verder kan een gebruiker eigen data toevoegen, zoals een validatie of commentaar bij een voorstel.

\subsection{Nuttige data die nog niet bestaat}
\subsubsection{Gebruikers}
Behalve paren en attributen is het in een collaboratieve omgeving handig om gebruikers bij te houden. Omdat het systeem in eerste instantie voor selecte groepen is, is een gebruikersnaam en een mailadres voldoende. Op die manier zou men kunnen vastleggen wie welk stukje data heeft toegevoegd of veranderd. Dit is nuttig om de oorzaak van problemen op te sporen als die zich voordoen en te beslissen wiens data voorrang krijgt als eenzelfde fragment door een andere persoon gewijzigd werd (zie synchronisatie).

\subsubsection{Geschiedenis}
Synchronisatie is een vereiste, hetgeen detectie en resolutie van conflicten impliceert. Detectie kan automatisch uitgevoerd worden door na te kijken of de huidige inhoud van een veld verschillend is. Resolutie daarentegen is niet altijd volledig te automatiseren. Om de gebruiker werk te besparen tijdens het synchroniseren moeten er toch zoveel mogelijk conflicten zonder manuele invoer opgelost worden. Cruciaal om dit te verwezenlijken is een geschiedenis van elk attribuut, dit wil zeggen een lijst die aanduidt op welk moment, door wie en naar welke waarde een attribuut is veranderd. Het bijhouden van een geschiedenis heeft ook andere voordelen, zo kan bijvoorbeeld de geschiedenis van een ``commentaar''-attribuut gebruikt worden als een manier om te corresponderen. De overgang van validaties van een paar (bijvoorbeeld \emph{niet geweten} $\rightarrow$ \emph{misschien} $\rightarrow$ \emph{niet correct} of  \emph{niet geweten} $\rightarrow$ \emph{misschien} $\rightarrow$ \emph{correct}) kan een nuttige bron van data zijn voor zowel mensen (statistieken) als computers (lerende algoritmen).

// merging: three way merge + context-informatie! (welke statussen beter zijn et cetera\ldots)
// een simplificatie die veel moeilijkheden vermijd bij het synchronizeren waar versiecontrolesystemen voor broncode hun tanden op breken, is dat in dit geval er geen verwijderingen kunnen gebeuren. Het huidige algoritme zal nooit een paar of attribuut verwijderen, er wordt ook geen geschiedenis bijgehouden voor de toevoeging of verwijdering van paren.

  
...relationeel van aard? (de geschiedenis wel, denormalisatie nodig voor NoSQL) => SQL database
Het Eigenschap-patroon (Property pattern)

Huidige situatie = Properties pattern? % ( http://steve-yegge.blogspot.com/2008/10/universal-design-pattern.html#redacted )

\section{Het oude systeem: XML-bestanden}
De fragmentparen en hun attributen worden door automatische herkenners opgeslagen in een XML bestand zoals in figuur~\ref{code:fragxml}. Dit formaat is uitermate geschikt voor het overbrengen van de data naar andere subsystemen en is leesbaar door een mens. Het is minder geschikt als een permanent opslag- en zoekformaat. Dat laatste is echter de manier waarop het nu gebruikt wordt, met trage applicaties tot gevolg. Om een redelijke snelheid te behouden tijdens het zoeken of sorteren moeten Griphos en Browsematches het XML bestand volledig in het werkgegheugen plaatsen. Voor een document met 50000 paren neemt dit reeds een goede 300MB in beslag voor paren met elk 11 attributen (zonder afbeeldingen of 3D-modellen). Dit extrapoleren naar een miljoen paren geeft 5GB aan vereist werkgeheugen, hetgeen bij de meeste systemen van vandaag leidt tot het wegschrijven van data naar het bestandsssyteem of eerder nog een geheugenallocatiefout.\\

Het inladen van dit XML bestand is tijdrovend. Tabel~\ref{table:matchesloadspeed} laat zien dat het inladen minutenlang kan duren. Griphos en Browsematches gebruiken dezelfde manier om data in te lezen, maar vertonen desalniettemin een groot verschil in laadtijd. De reden is dat Griphos meteen ook de afbeeldingen van elk fragment ophaalt. Meer nog, na het laden van 50000 paren neemt het Griphos proces 1.5GB RAM in beslag. Het lijkt erop dat dit geheugen niet vrijgemaakt noch herbruikt wordt, want een tweede keer de lijst met fragmentparen openen zorgde er op het testsysteem met 2GB RAM voor dat Griphos automatisch werd afgesloten wegens overallocatie. Browsematches (en de thesisapplicatie) laden de afbeeldingen pas wanneer ze eigenlijk nodig zijn en besparen op die manier veel geheugen. Het thesisproject gaat eigenlijk nog een stapje verder en laadt zelfs de fragmenten pas in wanneer ze nodig zijn, dit komt later aan bod. De reden dat de tijd dan niet gewoon 0 seconden aangeeft is omdat er steeds moet geteld worden hoeveel voorstellen er in totaal aanwezig zijn, dit kost bij sommige databaseimplementaties soms wat tijd.

\begin{table}[h]
	\rowcolors{2}{gray!50}{white}
	\begin{center}
		\begin{tabular}{|l|r|r|r|}
		    \rowcolor{gray!75}
		    \hline
		    & \textbf{Griphos} &  \textbf{Browsematches} & \textbf{Thesis} \\
		    \hline
		    \textbf{\textasciitilde 4000 paren laden} & 54 sec & 16 sec & 0 sec \\
		    \textbf{\textasciitilde 50000 paren laden} & 7 min 18 sec & 2 min 47 sec  & 0-3 sec* \\
		    \textbf{\textasciitilde 250000 paren laden} & niet getest & niet getest & 0-15 sec* \\
		    \hline
		\end{tabular}
	\end{center}
	\caption{Meting van de tijden die elk programma nodig heeft om een collectie paren in te laden}
	\label{table:matchesloadspeed}
\end{table}

* De tijden voor de oplossing van de thesis vari\"eren afhankelijk van de achterliggende databaseimplementatie en hoeveel data er nog in het geheugen geladen is. De maximale tijden kwamen bij tests enkel voor net na het opstarten van de database, de gemiddelde laadtijd is ongeveer 100 milliseconden.

Eenmaal geladen, is het niet zo eenvoudig om te navigeren in deze collectie

Als het echter aankomt op het aanwenden van de data om in te zoeken schiet het formaat tekort.

\lstinputlisting[float=h,label=code:fragxml,caption=Uittreksel van een fragmentpaarbestand (hier worden slechts 2 paren getoond)]{source/shortmatches.xml}

XML bestand volledig in het geheugen ingeladen ---> 300 MB voor 50000 paren, dus voor 2 miljoen paren\ldots

E\'en van de belangrijkste stappen op weg naar een collaboratieve applicatie, is het omvormen van\\

Het ontwerp van het database model en de implementatie details zijn groot genoeg om hun eigen hoofdstuk te verdienen, zij worden later in hoofdstuk [...] besproken.\\

\section{Alternatieven}
Het moet gratis zijn! (niet overdreven veel middelen in het thera project en ``enterprise'' oplossingen zijn vaak zeer prijzig) ==> open-source of tenminste gratis.

NoSQL (Cassandra, \ldots) http://arin.me/blog/wtf-is-a-supercolumn-cassandra-data-model --- http://nosql.mypopescu.com/post/573604395/tutorial-getting-started-with-cassandra (statische sort == slecht voor onze doeleinden)
SQL
XML database (oud) (zie wiki) XQuery, \ldots

XML dismissed -> volgende sectie is kiezen tussen SQL en NoSQL

Redenering!

\subsubsection{Paren}

\subsubsection{Attributen van paren}

\subsubsection{Complexe informatie die niet in een simpel attribuut past}

[uitleggen wat duplicates zijn, mss in ander hoofdstuk, mss in het stuk van de modules!]
[comments = correspondentie enzovoort]

\subsection{Pagination}

--> beperken van datatransfer
--> het niet steeds opnieuw vragen van hoeveel fragmenten er voldoen aan de criteria

\section{SQL of niet}
Er is dus een idee van wat er moet opgeslagen worden. Er is een grote vari\"eteit aan oplossingen mogelijk

De term NoSQL is eigenlijk niet zo goed gekozen (waarom?), maar wordt hier gebruikt omdat het de facto de standaard manier is om naar niet-relationele databases te verwijzen.

De zogenaamde ``NoSQL'' oplossingen schieten als paddenstoelen uit de grond. De meerderheid hiervan functioneren als sleutel-waarde opslag. Door de veelheid aan oplossingen en hun (relatief) jonge leeftijd is het niet evident een volledige vergelijking te maken. Nogal wat artikels over NoSQL oplossingen lijken zonder meer te evangeliseren, waardoor het moeilijk is een realistisch beeld te krijgen van hun specifieke sterktes en zwaktes.

NoSQL, lichte uitleg over NoSQL (Cassandra, \ldots)

De belofte van object serialisatie en schemaloze\footnote{In een traditionele relationele database is steeds een schema aanwezig dat precies specificeert wat een object (= rij in een tabel) is. Het is minder eenvoudig om de definitie van object aan te passen door bijvoorbeeld een attribuut (= kolom) toe te voegen, maar niet onmogelijk.} opslag is verleidelijk. De basisvereisten van dit thesisproject qua data gaan in eerste aanleg niet verder dan het opslaan/serialiseren van objecten (\emph{User}, \emph{FragmentPair}, \emph{History}). Het niet weten welke attributen er in de toekomst gaan belangrijk zijn lijkt goed te passen in het schemaloze aspect.\\

Een goede vergelijking van beide aanpakken kan als er gekeken wordt naar: kan de implementatie het soort operaties dat nu nodig is aan (op performante wijze? Kan de implementatie voorziene/niet-vooziene uitbreidingen aan (many-to-many)?\\


Vergelijking van NoSQL tegenover SQL oplossingen waar die toepasselijk zijn op de vereisten van het project

Berkeley DB, Tokyo Cabinet, CouchDB, MongoDB, Cassandra

Berkeley DB en Tokyo Cabinet zijn lightgewight en snelle oplossingen die er echter op vertrouwen dat alle data in het werkgeheugen past, de ondersteuning voor complexe vragen is ook bijna onbestaande. 

Er zijn vele argumenten over wat wel en niet geschikt is voor elk type database, waarvan een heel deel tegenstrijdig zijn. Uit dit kluwen komt eigenlijk slechts \'e\'en goede raad naar voren: probeer het zelf. Het is echter niet zo evident noch de beste besteding van tijd om twee verschillende maar complete subsystemen uit te bouwen om te zien welke de bovenhand haalt. Omdat er nog verschillende gebruiksscenarios niet gekend zijn werd er gekozen voor een SQL oplossing, waarvan de auteur meer zekerheid had dat het de taak en alle toekomstige taken aankon.

\rowcolors{2}{gray!50}{white}
\begin{center}
	\begin{tabular}{|l|r|r|r|}
	    \rowcolor{gray!75}
	    \hline
	    & \textbf{SQL} &  \textbf{NoSQL} & \textbf{Kyoto Cabinet} \\
	    \hline
	    \textbf{Portabiliteit/Uniformiteit} & Hoog (gestandaardiseerde taal) & Laag (veel verschillende talen) \\
	    \textbf{Ondersteuning in de toekomst} & Hoog & Laag (vele projecten die een onzekere levensduur hebben) \\
	    \hline
	\end{tabular}
\end{center}

Beide systemen bezitten voord- en nadelen. Wat echter duidelijk is, is dat bij NoSQL databases er op voorhand moet geweten zijn welk type query het meest frequent gaat zijn (Cassandra\ldots)
 Omdat er geen 100\% zekerheid is over de uitbreidingen die gaan komen aan het datamodel

Beide modellen voor dataopslag kunnen in principe dezelfde soort operaties aan, de verschillen bevinden zich in de gemakkelijkheid van uitvoeren en de performantie. Het is echter nog steeds niet evident welke uiteindelijk de optimale keuze is. De wereld van het databeheer bevindt zich in een stroomversnelling en op elk gegeven moment kan er een fantastische nieuwe oplossing uit de bus komen. Het valt op te merken dat op de manier er nu van een XML database gemigreerd zal worden naar een andere technologie, dit later openieuw kan gebeuren als er een duidelijk betere oplossing zich voordoet. Alle operaties op de objecten die in dit project reeds zijn ge\"implementeerd, zijn vanuit het perspectief van de gebruiker --- archeoloog of ontwikkelaar --- database-agnostisch. De schijnbare uitzondering op deze regel lijkt het filteren te zijn, waar een valide (vereenvoudigde\footnote{Hoewel er niet gecontroleerd wordt of er geen SQL subqueries in de modelfilter aanwezig zijn, is het gebruik ervan niet ondersteund (hoewel het voor de meeste SQL-databases geen probleem is deze filters uit te voeren).}) SQL WHERE gebruikt wordt om condities op de eigenschappen van paren te zetten. Deze zijn echter niet zo complex (zie \ref{code:sortingfiltering}) en dus gemakkelijk om te bouwen naar iets wat een willekeurige onderliggende database kan begrijpen. De WHERE-syntax kan dus ook, ook al verstaat de database het niet zonder meer. Indien dit niet genoeg is kunnen er gewoon meerdere filtertypes zijn, een SQL-type, een CouchDB-type, enzovoort. Een implementatie van een model zal dan zelf moeten uitzoeken hoe het alles moet vertalen of gewoon de filter weigeren.

Niet alleen is objectpersistentie nodig (de sterkte van key-value NoSQL databases), maar de manier om de objecten te vinden moet zo flexibel mogelijk zijn. NoSQL databases bieden verschillende hulpmiddelen aan om dit te doen (MapReduceCombine, CouchDB Views), \ldots 

Een lichte voorkeur voor de SQL oplossingen is niet verwonderlijk. Er zijn verschillende implementaties met uniforme taal De auteur is reeds bekend met de werking en als men moeilijk kan voorspellen wat voor een queries zullen nodig zijn (data-mining) blijken SQL databases sneller te zijn. SQL lijkt gemakkelijker te gebruiken.\\

Een van de belangrijkste voordelen van NoSQL type databases is de ingebouwde schaleerbaarheid over meerdere machines. Het kost werkelijk geen moeite om deze implementaties over meerdere machines te laten samenwerken en hun performantie gaat zo goed als lineair met het aantal machines omhoog. Zullen de datavereisten van het thera project ooit zo groot worden dat een enkele krachtige machine niet meer voldoet? Op het eerste zicht lijkt het van niet, hedendaagse . Daarenboven is het daarom niet onmogelijk om SQL databases in een cluster te laten werken, gratis open-source implementaties als MySQL en PostgreSQL hebben beide ondersteuning voor clustering~\cite{postgrescluster, mysqlcluster}. Het verschil is de eenvoudigheid waarmee dit uit te bouwen is.\\

De syntax van NoSQL is niet simpeler, ook niet voor de specifieke queries om paren op te halen (zie infographic). Een andere vaststelling is dat de bibliotheek die door het thera project en dus ook dit thesisproject gebruikt wordt, Qt, ingebouwde ondersteuning heeft voor SQL databases. Het nalezen van artikels en conversaties op het internet en private correspondentie met ervaringsdeskundigen heeft geleerd dat NoSQL databases hun snelheid laten blijken bij veel transacties en een beperkte klasse van verzoeken, die moet echter op voorhand gekend zijn\cite{cassandradatamodel}. In essentie moet de data ingedeeld worden zodat het soort verzoeken dat verwacht wordt snel kan afgehandeld worden [REFERENTIE]. Hoewel een deel van de verzoeken die nodig zijn in dit project daadwerkelijk gekend zijn, is \'e\'en van de uitgangspunten net dat er een grote vari\"eteit aan verzoeken moet mogelijk zijn.  

Om al de redenen is er gekozen om het stabiele SQL pad te kiezen, met de deur open naar eventuele alternatieven wanneer die een overtuigend voordeel bieden.

[figuur mogelijke NoSQL uitbreiding, TokyoMatchModel, CouchDBModel,\ldots] 

Een voordeel van SQL systemen is dat er eenvoudig tussen verschillende imlementaties kan gemigreerd worden als de vereisten veranderen. Een broekzakversie die bij momenten geen connectie met het internet heeft kan op het compacte SQLite vertrouwen terwijl een krachtige server een heel grote database kan draaien met MySQL/PostgreSQL/Oracle/\ldots. Al deze verschillende implementaties kunnen met dezelfde subsystemen aangesproken worden. Bij vele NoSQL systemen is dit soort flexibiliteit niet mogelijk (een uitzondering is bijvoorbeeld http://fallabs.com/kyotocabinet/).\\

http://stackoverflow.com/questions/5438500/example-of-a-task-that-a-nosql-database-cant-handle-if-any (auto-sync\ldots)
http://stackoverflow.com/questions/2403174/is-there-any-nosql-database-as-simple-as-sqlite
Supergoede argumenten voor en tegen NoSQL/SQL: http://buytaert.net/nosql-and-sql

Dat gezegd zijnde, behalve de query interface

\section{Eerste iteratie}

Zonder geschiedenis/gebruikers

Geschiedenis invullen kan met triggers, maar dit is op het moment niet zo, op die manier heeft een applicatie de vrijheid om veranderingen aan te brengen zonder de geschiedenis te vervuilen. Als dit een slechte beslissing blijkt te zijn kunnen de triggers wel aangezet worden.

\subsection{De database laag}

UML-achtig schema met de Database, het MatchModel, SQLFragmentConf, de interopabiliteit!!!!

----> hieruit vloeit de requirement van een matches tabel en attributen

Om compatibel te zijn met het grotere thera project moeten alle paren hun eigen eigenschappen kunnen aanpassen. Dit wordt in de SQL versie opgelost door te eisen dat valide paren steeds geconstrueerd worden met een referentie naar de database waar ze uit komen. Op deze manier wordt de meest eenvoudige implementatie van een paar ongeveer zo geschreven:

\lstinputlisting[label=fragmentpair1,caption=Een versimpelde interface voor een FragmentPair en diens implementatie met directe verzoeken aan de database, language=C++]{source/pairinterface.cpp}

\subsection{Een grote tabel of vele verschillende}

Om de attributen van een paar te modelleren, kan er gekozen worden tussen deze allemaal op te slaan in de hoofdtabel (\emph{matches}) of voor elk attribuut een andere tabel aan te maken.

Indien het mogelijk zou zijn dat een (grote) subset van paren een attribuut gewoonweg niet bezit, kan het 

Hybride aanpak:

History altijd in verschillende tabellen

%http://stackoverflow.com/questions/4056093/what-are-the-disadvantages-of-using-a-key-value-table-over-nullable-columns-or
%http://stackoverflow.com/questions/695752/product-table-many-kinds-of-product-each-product-has-many-parameters
%http://en.wikipedia.org/wiki/Entity-Attribute-Value_model

Zijn de meeste attributen gedefini\"eerd voor alle paren?

Voordelen denormalizatie: de index wordt niet ettelijke keren redundant herhaald
Voordelen normalizatie

Simpele attributen, database query voor elk attribuut (dit bleef in stand tot er op een externe server geprobeerd werd)
Directe write-through is nooit een echt probleem omdat dit minder vaak gebeurt, op minder fragmenenten\ldots men zal veel vaker van venster wisselen dan ineens alle statussen omgooien

Dependency analyse voor filters

\subsection{Complexe informatie voorstellen als een attribuut}
Meta-attributen, VIEWs ---> duplicates (conflicts?)

\subsection{Geschiedenis}
De geschiedenis van een attribuut bijhouden heeft twee voordelen: het is nodig voor samenvoegalgoritmen + het is een weer een bron van data (welke paren zijn veel van status veranderd?) + sommige attributen worden op die manier nuttiger, 

\section{Verschillende database systemen}

\subsection{SQLite}
Aan de start van het project leek

\subsection{MySQL}
MySQL werd eerst gekozen wegens de alomtegenwoordige

\subsection{De late toevoeging van PostgreSQL}
Dankzij de flexibiliteit die nodig was om zowel SQLite als MySQL het grootst mogelijke deel van de code te laten delen

\section{Data mining}
Maak ketting van (yes+maybe) ---> zoek naar alle buren binnen 1 hop (algoritme is geschreven)

\section{Externe database traag, interne database snel}
Het grote probleem dat duidelijk werd na het rechtstreeks werken met externe databases, is dat zelfs op een lokaal netwerk de verzoeken een zeer grote vertraging opleverden.\\

De redenen hiervoor waren natuurlijk de vele queries die elk object kan versturen om zijn attributen op te halen of te veranderen.\\

Nog een groot voordeel bij het maken van dergelijk systeem is dat er gebruik kan gemaakt worden van niet-desctructieve veranderingen. Op het einde van een werksessie kan men de volledige
lokale database ofwel in de hoofd database invoegen ofwel gewoon verwijderen.\\

Het design moest transparant zijn, zowel een rechstreekse connectie als een gebufferde connectie zouden voor de eindgebruiker en de ontwikkelaar aan de buitenkant hetzelfde lijken.\\

Idee: Vervang object ID met object hash voor snellere match-to-match identificatie zonder een globala id nodig te hebben

\section{Slimme client of slimme server?}
Het programma bevat alle nodige functionaliteit en de server is in dat opzicht gewoon de specifieke database waarnaar het een verwijzing heeft.

Voor het gebruik op kleine apparaten (zoals tablets) zijn de mogelijkheden voor computationeel zware activiteiten niet zo uitgebreid. Daarenboven moet ervoor gezorgd worden dat de batterijduur
van dit apparaat niet te hard beknot wordt door het gebruik van de applicatie. Daarom zal de transitie naar zeer mobiele platformen enkel mogelijk worden indien er een simpele client applicatie kan
ontwikkeld worden die op de server vertrouwd om de juiste berekeningen te maken.\\

In de thesisperiode is geen tijd gevonden om dit concept uit te werken maar er zijn wel plannen gemaakt waardoor dergelijke
applicatie op een effici\"ente manier zou kunnen ontwikkeld worden. De gebruikersinterface Tangerine voert alle functies met betrekking tot de achterliggende data uit met behulp van een grote bibliotheek
van klassen die op zich niets met de interface te maken hebben. Er zou dus een alternatief programma gebouwd kunnen worden die in plaats van met een grafische interface kan bediend worden
via een zelfgemaakt protocol. Dit soort ontwerp wordt vaak aangetroffen in de UNIX wereld, waar er een enkele server-variant van een programma bestaat en meerdere mogelijke clients die ervan gebruik maken.
Het bekendste is misschien wel de X server.

Materialized views!!!

Model change batching!!! (startBatch/endBatch)

Incompatibiliteiten tussen *SQL

Analyse van datatoegangspatroon: vooral SELECT, ORDER BY, GROUP BY ---->
veelvuldig gebruik van indexes

Metadata preloading (na het testen van de snelheid van het ophalen van metadata
over een internetconnectie, werd besloten om\ldots)

De gevaren van een stale cache! (en ook de gevaren van een stale window, ook
een soort cache)). Oplossing altijd een request sturen om te dubbelchecken?!
---> te traag

optimize voor fast reads -----> inserts kunnen lokaal gedaan worden, updates hopelijk niet zo veel of lokaal

Recheck om de X seconden: doenbaar indien materialized views met indices!
(eigenlijke window query < 1 msec). Systeem zou 1000'en gebruikers kunnen
ondersteunen, zeker met een grotere cache

Cache = write-through

Vele discplines van de computerwetenschappen

% http://en.wikipedia.org/wiki/Three-way_merge#Three-way_merge <--- (bij design\ldots)

dynamisch zoekopdrachten genereren

SQLite formaat maakt database gemakkelijk te delen (USB-stick) <-- geen internet
connectie

Detecteren van imcompatibiliteiten EN veranderingen met reguliere
expressies:

Aanpaken voor syncrhonizatie:
moeilijk: changelogging functionaliteit, triggers, \ldots (mogelijk maar
moeilijk cross-db en error-prone)
``gemakkelijk'': Maak gewone queries zeer goedkoop

Vereisten: Scaleerbaarheid (XML schaalt NIET)
Voordeel van XML is echt wel dat het een mooi outputformaat is voor
matchers\ldots import capaciteit voorzien ==> import naar temp SQLite db en
merge

Ondersteuning voor ``meerdere snelheden'', minder modules hebben is niet erg,
elke additie is uitbreiding. Zo ook minder/geen versie van db-schema problemen

dependency scanning!

\section{Benchmarking}
De query cache staat af
Om de variabiliteit van de filesystem (nederlands) cache een beetje buiten spel te zetten zijn al deze routines ``opgewarmd''

Pagination
Late row lookup

Ideaal = minder dan een seconden voor elke gegeven query vanuit een gebruikersinteractie standpunt (System Response Time and User Satisfaction pagina 5)

Effect van DB configuratie (veel geheugen\ldots)

Suggested workaround voor het text probleem -> sphinx, restrict fragment names (niet ZO gemakkelijk), string + nummer
Suggested workaround voor het indexing probleem (zoals gezien voor status IN (\ldots)) -> force een index?! dunno\ldots hij pakt in ieder geval de verkeerde!

'High Performance MySQL', Second Edition, O'REILLY, ISBN: 978-0-596-10171-8
MySQL Reference Manual for version 5.1

http://nlp.stanford.edu/IR-book/html/htmledition/permuterm-indexes-1.html (dit is hoe wildspeed werkt) (LIKE performance lijkt niet zo slecht in Postgres, het is de sorting eerder\ldots)
http://www.slideshare.net/techdude/how-to-kill-mysql-performance
http://stackoverflow.com/questions/1540590/how-to-speed-up-like-operation-in-sql-postgres-preferably <--- use trigrams (fail), MAAR BETER IN 9.1 (future research)
https://cgsrv1.arrc.csiro.au/blog/2010/06/23/materializedindexed-views-for-postgresql/
\chapter{Synchronisatie}
\label{hoofdstuk:synchronisatie}
Het thera project bestaat al een tijdje en er zijn reeds vele XML-bestanden met validaties van voorstellen in omloop. Ondanks de eventuele overgang van het oude naar het nieuwe systeem is het belangrijk dat deze resultaten niet verloren gaan, het heeft immers veel werk gekost om ze te maken. Als bestanden verschillende paren bevatten --- omdat ze bijvoorbeeld door een ander identificatiealgoritme zijn gegenereerd --- maar over eenzelfde opgraving gaan, is het ook nuttig om deze te consolideren. Indien er geen internetverbinding is of even geen storing van anderen wil hebben, kan een lokale kopie ge\"extraheerd en later terug ge\"importeerd worden. Dit alles kan gedaan worden met de synchronisatiecomponenten van dit project.

\section{Opzet}
Om te synchroniseren selecteert men een meester- en een slaaf-database, de conventie is dat de slaaf door de meester zal geabsorbeerd worden. Na het proces heeft de meester alle gewenste veranderingen en blijft de slaaf onveranderd over. Er zijn verschillende fasen (voorlopig 3): \textbf{Gebruikers}, \textbf{Paren} en \textbf{Attributen}. In elke fase krijgt de gebruiker een scherm te zien met daarin de verschillen tussen beide databases, het biedt de mogelijkheid om aanpassingen te maken alvorens naar de volgende stap te gaan. Het is belangrijk gebruikers keuzes te laten maken waarin ze ge\"interesseerd zijn en ze niet lastig te vallen met keuzes die ze niet willen maken, beide zaken negeren leidt tot frustraties~\cite{Joel2001}. De algoritmes proberen daarom steeds een standaardkeuze in te vullen aan de hand van heuristieken, deze kunnen indien gewenst verworpen worden. Als er door gebrek aan informatie geen automatische belissing kan genomen worden, kan er niet naar de volgende fase worden overgegaan zonder alle conflicten eerst manueel op te lossen. Figuren~\ref{fig:merge-attrib} en~\ref{fig:merge-choose} tonen hoe deze stappen eruitzien en figuur~\ref{fig:merging-overview} geeft een schematische kijk op het systeem.\\

\begin{figure}[ht]
	\begin{center}
		\includegraphics[width=1.0\columnwidth]{images/merge-action-1-small.png}
		\caption{Een uittreksel van de attributen-fase van het synchronisatieproces. ``No Action''  duidt aan dat er geen automatische resolutie toegepast is geweest, hierop dubbelklikken geeft een keuzescherm weer (rechts).}
		\label{fig:merge-attrib}
	\end{center}
\end{figure}

\begin{figure}[ht]
	\begin{center}
		\includegraphics[width=0.5\columnwidth]{images/merge-action-2-small.png}
		\caption{Elk type conflict laat een aantal acties toe om het op te lossen. Deze figuur stelt een attribuut-conflict voor.}
		\label{fig:merge-choose}
	\end{center}
\end{figure}

Het gros van het onderzoek op het vlak van samenvoegingsalgoritmen gaat over het combineren van tekstbestanden, hier werden reeds vele innovatieve aanpakken voor ontwikkeld\footnote{Arbitraire tekst is moeilijk to synchroniseren omwille van de vele vrijheidsgraden, een tekstbestand kan op ontelbare manieren veranderen en het is niet steeds triviaal om te zien hoe twee conflicterende aanpassingen in elkaar moeten gevoegd worden}. Gelukkig vallen vele problemen weg door de eenvoudigere (minder vrije) structuur van de data in dit project. Voor de attributen wordt een geschiedenis bijgehouden. De algoritmen gebruikt voor elke fase worden in de volgende secties uit de doeken gedaan.

\subsection{Paren}
In deze stap worden de voorstellen zonder attributen uit beide databases vergeleken en gesynchroniseerd. Deze objecten bevatten een identificatiecode, twee namen van fragmenten en een 3D-transformatiematrix. De identificatiecode is echter geen uniek gegeven over alle databeses, eerder een manier om effici\"ent naar het object te verwijzen binnen eenzelfde database. Het is echter duidelijk dat als twee voorstellen uit dezelfde fragmenten bestaan en dezelfde transformatie hebben, ze identiek kunnen geacht worden. De aanwezigheid van afrondingsfouten en duplicaten is problematisch. Duplicaten zullen per definitie een gelijkaardige transformatiematrix hebben, hoe ze te onderscheiden van afrondingsfouten? De matrix van een voorstel uit de slaaf-database wordt afgetrokken van de matrix van elk voorstel uit de meester-database dat uit dezelfde fragmenten bestaat, hiervan wordt vervolgens de Frobenius norm\footnote{De euclidische norm van de vector die ontstaat als men de rijen of kolommen van een matrix achter elkaar plaatst.} genomen. Hieruit ontstaat een rij van getallen die aangeven hoe gelijkaardig het voorstel is aan elk van de mogelijke overeenkomstige voorstellen uit de meester-database. Het kleinste getal geeft het meest overeenkomstige paar aan. Indien dit kleinste getal een ordegrootte (10x) kleiner is dan het maximale verschil tussen dit overeenkomstig paar en diens duplicaten, wordt aangenomen dat het de paren in kwestie identiek zijn. Indien niet wordt het paar uit de slaaf-database als een nieuwe paar ingevoerd.\\

De volledige voorstellenlijst van beide databases moet dus nagekeken worden. Eenmaal beiden in het geheugen geladen zijn verloopt het vergelijkingsproces vrij snel (er wordt een hashmap aangemaakt om in constante tijd te kunnen zien welke voorstellen uit dezelfde fragmenten bestaan). Dit wil natuurlijk wel zeggen dat het synchroniseren van pure paren op zich over het internet niet werkbaar is met een database van miljoenen elementen. Gelukkig gebeurt het niet vaak dat onderzoekers paren toevoegen.\\

Paren die overeenkomen krijgen automatisch de actie ``map id'' toegewezen, waardoor latere fasen weten dat het object in kwestie verhuisd is. Een paar dat niet gevonden werd in de meester-database wordt niet automatisch ingevoerd, maar als een conflict weergegeven. De gebruiker kan gemakkelijk kiezen om dit toch te doen door ``assign new id'' als actie toe te kennen en aan te vinken om dit te doen voor alle gelijkaardige conflicten (zoals in figuur~\ref{fig:merge-choose}). Het alternatief is ``don't merge'', wat ervoor zorgt dat de nieuwe paren niet toegevoegd worden (en genegeerd in de volgende stappen).

\subsection{Attributen}
Het samenvoegen van attributen verloopt op een andere manier. Het is gemakkelijk te detecteren wanneer zich een conflict voordoet, namelijk als de waarden verschillen. Dit conflict oplossen gaat echter niet automatisch, als een \emph{commentaar}-attribuut bijvoorbeeld twee verschillende teksten bevat, welke is dan de juiste? Misschien de meest recente, zeker als blijkt dat een vorige waarde van de meest recente gelijk is aan de minder recente (gemeenschappelijk ouder). Om automatische resolutie te ondersteunen moet er dus een geschiedenislijst bijgehouden worden. Daarnaast kan nog gekeken worden naar de semantische inhoud van het attribuut om eventueel een oordeel te vellen als de geschiedenismethode faalt.

\subsubsection{Gemeenschappelijke ouder}
Indien beide databases voor een bepaald attribuut een stukje geschiedenis delen, is er een kans dat het conflict automatisch opgelost kan worden. Stel dat in een kopie van een database een attribuut veranderd en ditzelfde attribuut is niet aangeraakt in de originele database. In dat geval kan de nieuwe waarde zonder meer overgenomen worden bij synchronisatie. Het kijken naar een gemeenschappelijke ouder om te synchronizeren wordt in de literatuur \emph{3-way merging}~\cite{sync:diff3} genoemd. De zojuist beschreven situatie komt overeen met het meest linkse vak in figuur~\ref{fig:okay-3-way}. Het is de meest voorkomende want uit de grote hoeveelheid paren is de kans klein dat exact dezelfde (in een korte tijdsspanne) bewerkt worden. In het geval dat dit wel zou gebeuren --- het middelste vakje in de figuur --- kan er niet op die manier gesynchroniseerd worden. Hetzelfde doet zich voor als de beide databases geen geschiedenis delen, deze situate wordt voorgesteld door het meest rechtse vakje. Samenvattend kan er dus niet automatisch gehandeld worden als de geschiedenis divergeert ofwel niet gedeeld is.

\subsubsection{Context}
Toch zijn er situaties denkbaar waar er toch een oplossing is. Stel dat de validatiestatus van een bepaald voorstel ``niet geweten'' is en de database wordt gesplitst in database A en B. In database A wordt vervolgens de status op ``misschien'' gezet en in B op ``correct''. Als de synchronisatie afweet van de semantiek van statussen, zal het correct kunnen afleiden dat ``correct'' een krachtiger status is en dit verkiezen. Voor commentaren kan bijvoorbeeld gelden dat het meest recente de bovenhand krijgt. Dit zijn contextbeslissinggen, waarbij de component op de hoogte moet zijn van de betekenis achter de waarden. Dit soort acties kan over het algemeen niet overgezet worden van het \'ene attribuut op het andere, hetgeen aparte routines voor elk contextgevoelig attribuut veronderstelt.\\

Er zijn ook attributen waar dit soort beslissingen niet eenvoudig kunnen genomen worden (bvb. duplicaten), of waarvoor de routines nog niet gemaakt zijn. In dit laatste geval kan er geen automatische resolutie plaatsvinden en moet de gebruiker ingrijpen alvorens verder te gaan.

\begin{figure}[ht]
	\begin{center}
		\includegraphics[width=1.0\columnwidth]{images/3-way-merge.png}
		\caption{Situaties waarin automatische resolutie gebaseerd op geschiedenis mogelijk is (links) en waarin niet (rechts)}
		\label{fig:okay-3-way}
	\end{center}
\end{figure}
\chapter{Modules}
\label{modules}

Hier komen de modules!!!!

Pure grafische plugins die op andere plugins vertrouwen voor data-selectie: Om de werkbaarheid van dit systeem uit te testen werd een voorbeeldmodule ontwikkeld die alle huidige paren in een grafe plaatst en deze met een ontwarringsalgoritme probeert te plaatsen, zodat men een globaler beeld kan krijgen van de huidige selectie.

\section{MatchTileView}

Conflicten, verzamelingsreductie door conflicten (hoeveel percentage kan zo uitgeruled worden?) ---> space-filling curve

zoeken naar buren, niet conflicterende buren, duplicaten aanduiden, icoonbalken voor commentaren, DetailView, blabla

\section{Proof of concept: GraphView}
\chapter{Besluit}
\label{besluit}
Nog geen besluit

%%% Local Variables: 
%%% mode: latex
%%% TeX-master: "masterproef"
%%% End: 

\chapter{Toekomstig werk}
\label{toekomst}

\section{Nieuwe platformen}
Met dit project is ook een verdere stap gezet naar mobiele toepassingen. Het data-luik staat bijvoorbeeld toe om applicaties voor tablets te schrijven die beroep kunnen doen op een externe database om een groot deel van het zware werk over te nemen. Dit soort programma's kunnen het manueel verifi\"eren van paren sneller en aangenamer maken. De huidige werkwijze voor finale verificatie is als volgt: nadat er een resem waarschijnlijke paren zijn ge\"identificeerd, kan men de kisten met fragmenten uit de opslagruimte halen en nakijken welke er \'echt passen. Gezien de grote hoeveelheid brokstukken is het niet mogelijk om ze allemaal bij de hand te houden. Daardoor duurt het altijd even voor de gewenste fragmenten gevonden worden. In het slechtste geval wordt er gewerkt op een (krachtige) desktop. Hierdoor is het nodig om ofwel de namen en locaties van de fragmenten te onthouden, ofwel een heleboel afbeeldingen af te drukken. Na het fysisch testen van de fragmenten moet men dan terug naar de desktop om de bevindingen in te geven. Gelukkig behoort een laptop ook tot de mogelijkheden hoewel applicaties als Browsematches en Griphos niet bepaald licht zijn. Het performantieprobleem wordt natuurlijk reeds een deel verholpen door het invoegen van een externe database. Ook kan men nu gemakkelijker met meerdere mensen en laptops tegelijkertijd werken aan de validaties wegens de automatische synchronisatie. Een stap verder zou zijn om een tablet te gebruiken. Er zijn reeds in het verleden experimenten geweest binnen het thera project om aanraakgevoelige omgevingen te maken en de hoop is dat tesamen met de resultaten van deze thesis er in de toekomst iets concreets van gemaakt kan worden.\\

\section{Verder innoveren}
Dynamische herberekening met procesverloop. In de realiteit wordt de finale conclusie over een voorstel gemaakt als men de fysieke brokstukken op de aangegeven plaats aan elkaar zet en ze ``klikken''. Door erosie is de klik op zich niet altijd 100\% vast waardoor het gebeurt dat zelfs met de fragmenten in de hand een amateur nog altijd geen sluitende conclusie kan geven. Evenwel is het \'e\'en van de krachtigste controles en zijn er niet veel voorstellen die twijfelgevallen blijven. Misschien is (een deel van de) oplossing om de weergave dynamisch te maken en zo te pogen de stabiliteit van het voorstel te meten. Uit ervaring is gebleken dat mensen snel kunnen opmerken wanneer een voorstel \textbf{potentieel} heeft, maar niet in een oogopslag kunnen beslissen over de juistheid. Vaak blijft er zelfs na een betere kijk op de statische virtuele beelden nog genoeg twijfel over: noch de gewone visualisatie nog de beschikbare alternatieven geven uitsluitsel. In acht nemend dat de automatische passers van het thera project rekening moeten houden met computationele effici\"entie en daarom niet te granulair kunnen zoeken~\cite{Brown2008}, is er dikwijls nog wat ruimte voor verbetering. Een manier om dit op te merken is dat de doorsneden van paren die correct blijken te zijn vaak nog delen vertonen waar de volumes van beide brokstukken elkaar snijden. De erosie die plaatsvindt op de fresco's die onderzocht worden is echter puur subtractief, er zet zich geen extra materiaal vast op de brokstukken. Er is dus een goede kans dat een juist voorstel verder kan geoptimaliseerd worden.\\

Interessante voostellen zouden daarom kunnen gebruik maken van een adaptieve versie van de passer die, beginnend van de originele positie, heel granulair kan zoeken. Wat voor miljoenen combinaties te kostelijk is, is voor een enkel voorstel slechts een peulenschil. Dit proces van iteratieve verbetering kan getoond worden aan de gebruiker, door een tijdsverloop van de doorsnede van het paar tijdens de optimalisatie weer te geven. Het komen en gaan van gebieden waar de geometrie van de fragmenten snijdt en waar ze verder uit elkaar gaat kan op zich reeds informatief zijn, maar het eindpunt is cruciaal. Na een volledige optimalisatie kan er een beeld getoond worden van lichte verschuivingen van het optimale punt. De hypothese is dat --- als de resolutie van de 3D-opname voldoende is --- de doorsnede van een paar dat in de realiteit klikt herkenbare patronen zal vertonen als het verschoven wordt. Hierover valt te verstaan dat er weerstand zal zijn in de vorm van snijdende geometrie. Er wordt hier abstractie gemaakt van enkele belangrijke implementatiedetails, zoals de richting waarin verschoven wordt. Dit kan gaan van eenvoudigweg de richting parallel kiezen aan de scheidingslijn tussen de uiterste punten, tot een complexe fysische simulatie die de fragmenten als rigide lichamen voorstelt en een combinatie van drukkende en schuivende krachten uitoefent. In de eerste vorm kan de weerstand ``gemeten'' worden door de gebieden van snijdende geometrie te sommeren. De tweede vorm laat toe de frictie onrechtstreeks te meten aan de hand van bijvoorbeeld verwachte tegenover re\"ele snelheid in een bepaalde richting, of beter nog de kracht die nodig was om een verschuiving te krijgen. Het is voorlopig onduidelijk of de toegenomen complexiteit van een fysische simulatie in verhouding staat met de resultaten.\\

Natuurlijk kan een aspect van deze beweging ook opgenomen worden als een karakteristiek in lerende algoritmen, men zou het ``stabiliteit'' kunnen noemen. Dit is een voorbeeld van een eigenschap die als discriminator kan gebruikt worden op voorstellen die reeds goed of interessant te noemen zijn, een slecht voorstel zou namelijk veel stabiliteit kunnen vertonen door de vele intersecties die zelfs na optimalisatie overblijven (afhankelijk van de precieze methode waarop men stabiliteit meet). Er zijn ook goede voorstellen waar de stabiliteit laag kan zijn doordat het materiaal teveel is afgevlakt of de breuk gewoonweg te rechtlijnig is. De stabiliteit lijkt dus geen gouden graal te zijn. Daarom wordt hoe langer hoe duidelijker dat net zoals een mens meerdere eigenschappen in rekening brengt bij het oordelen, een computer dit ook moet doen.


% Indien er bijlagen zijn:
\appendixpage*  
\appendix
\chapter{Extra afbeeldingen \& schema's}
\label{app:A}
\section{Omzetting SQL naar NoSQL}
\begin{center}
\hvFloat[
%floatPos=h,
nonFloat=true,
capWidth=1,%
capPos=b,%
rotAngle=0,%
objectPos=c%
]{figure}{\includegraphics[width=0.6\textheight]{images/SQL-to-MongoDB.pdf}}{Sommige NoSQL systemen --- zoals MongoDB --- beschikken zoals te zien valt in deze afbeelding over dezelfde analytische kracht als een SQL database. De performantiekarakteristieken zijn echter verschillend. Afbeelding gebruikt met toestemming~\cite{sqlnosql}}{fig:sqlmongo}
\end{center}
%\clearpage
\pagebreak

%\includegraphics[width=0.8\textheight]{images/SQL-to-MongoDB.pdf}
%\includepdf{images/SQL-to-MongoDB.pdf}

\section{Overzicht van de applicatie}
\hvFloat[
%floatPos=h,
nonFloat=true,
capWidth=1,%
capPos=b,%
rotAngle=90,%
objectPos=c%
]{figure}{\includegraphics[width=0.95\textheight]{images/Bigtang.png}}{Een ruw overzicht van de componenten in het project en hoe het interageert met het reeds bestaande systeem.}{fig:tangbig}
\clearpage

\section{Detail van de synchronisatiestructuur}
\hvFloat[
%floatPos=h,
nonFloat=true,
capWidth=1,%
capPos=b,%
rotAngle=0,%
objectPos=c%
]{figure}{\includegraphics[width=\textwidth]{images/Overview-merging.png}}{Het synchronisatie-subsysteem, de componenten die overeenkomen met de stappen zijn aangeduid.}{fig:merging-overview}
\chapter{Poster}
\label{app:poster}

\includepdf{poster/poster.pdf}

\backmatter
%\bibliographystyle{abbrv}
\bibliographystyle{ieeetr}
\bibliography{referenties}

\end{document}
